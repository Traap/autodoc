% ------------------------------------------------------------------------------
% Design History File index section.
% ------------------------------------------------------------------------------
% Define column names use by csvreader for the design history file..
\csvnames{dhfNames}%
{1=\pj,2=\pv,3=\ty,4=\dn,5=\ver,6=\ttl,7=\pkg,8=\dra,9=\drb,10=\drc,11=\drd,12=\dre,13=\ai}%

% Define a table style for the design history file csvreader.
\csvstyle{dhfStyle}{%
  longtable=|C{0.5cm}|L{3cm}|C{1.5cm}|L{9cm}|
  ,table head=\hline \rowcolor{cyan}\# & Document & Version & Title \\ \hline \hline \endhead
  ,late after line=\\\hline
  ,dhfNames
}

% ------------------------------------------------------------------------------
% Define commands to select rows from the design history file.
% Parameters:
% #1 - relative path to design history file.
% #2 - project column value 
% #3 - project version column value
% #4 - document type or AIor AI  column value
\newcommand{\dhfList}[4]{\dhfQuery{#1}{\pj}{#2}{\pv}{#3}{\ty}{#4}}%
\newcommand{\aiList}[4] {\dhfQuery{#1}{\pj}{#2}{\pv}{#3}{\ai}{#4}}%

% ------------------------------------------------------------------------------
% Define a command to query rows from the design history file.
% Parameters:
% #1 - relative path to design history file.
%
% #2 - first column name to match
% #3 - first column value to match 
%
% #4 - second column name to match
% #5 - second column value to match 
%
% #6 - third column name to match
% #7 - third column value to match 
\newcommand{\dhfQuery}[7]{%
  \IfFileExists{#1}{%
    \begin{center}%
      \csvreader[dhfStyle, separator=pipe, respect all,%
        filter={\equal{#2}{#3} \and \equal{#4}{#5} \and \equal{#6}{#7}}%
        ]%
        {#1}{}{\thecsvrow&\dn&\ver&\ttl}%
    \end{center}%
  }%
  {\todo[inline]{#1 was not found.}%
  }%
}%

% ------------------------------------------------------------------------------
