% ------------------------------------------------------------------------------
% These are the parameters Results Synthesis uses when calling the
% includeTwoColumnTable command. These parameters maybe overridden in  
% data/additional-layout.tex for the document you are writing.
% 
% Note:  Do not reset these parameters in this file.
% ------------------------------------------------------------------------------
\newcommand{\ttSynFile}{test-output/resultsSynthesis.csv}%
\newcommand{\ttSynColAHdr}{Case Type}%
\newcommand{\ttSynColBHdr}{Result}%
\newcommand{\ttSynSecName}{Results Synthesis}%
\newcommand{\ttSynSecHeading}{resultssynthesis}%
\newcommand{\ttSynSecText}{%
  This section compiles the counts for Total Test Cases, Passed Test Cases, and
  Failed Test Cases, and determines if the test plan is COMPLIANT or
  NON-COMPLIANT.
}%

% ------------------------------------------------------------------------------
% These are the parameters Test Environment uses when calling the
% includeTwoColumnTable command. These parameters maybe overridden in  
% data/additional-layout.tex for the document you are writing.
% 
% Note:  Do not reset these parameters in this file.
% ------------------------------------------------------------------------------
\newcommand{\ttEnvFile}{test-output/envInfo.csv}%
\newcommand{\ttEnvColAHdr}{Name}%
\newcommand{\ttEnvColBHdr}{Value}%
\newcommand{\ttEnvSecName}{Test Environment Variables}%
\newcommand{\ttEnvSecHeading}{testenvironment}%
\newcommand{\ttEnvSecText}{%
  This describes the state of the system environment used for testing.  The
  following values are generated from config files and reading the ENV on the
  client system.
}%

% ------------------------------------------------------------------------------
% This set of commands are defined with default values.  They are redefined to
% the values used as parameters to command includeTwoColumnTable
% ------------------------------------------------------------------------------
\newcommand{\dataFile}{}%
\newcommand{\dataName}{}%
\newcommand{\dataValue}{}%
\newcommand{\dataSectionName}{}%
\newcommand{\dataSectionTitle}{}%
\newcommand{\dataSectionBody}{}%

% ------------------------------------------------------------------------------
% Redefine data values.
% ------------------------------------------------------------------------------
\newcommand{\redefineDataParameters}[6]{%
  \renewcommand{\dataFile}{#1}%
  \renewcommand{\dataName}{#2}%
  \renewcommand{\dataValue}{#3}%
  \renewcommand{\dataSectionName}{#4}%
  \renewcommand{\dataSectionTitle}{#5}%
  \renewcommand{\dataSectionBody}{#6}%
}%

% ------------------------------------------------------------------------------
% Define csvnames and csvstyle
% ------------------------------------------------------------------------------
\csvnames{dataNames}{1=\dataName, 2=\dataValue}%

\csvstyle{dataStyle}{%
  longtable=|C{1.0cm}|L{7.0cm}|L{7.0cm}|%
  ,respect all%
  ,table head=\hline%
    \rowcolor{fkblue}%
    \# & \fkHdrRow{\dataName} & \fkHdrRow{\dataValue}\ER%
  \endhead%
  ,late after line=\ER%
  ,table foot=\hline%
  ,dataNames%
}%

% ------------------------------------------------------------------------------
% includeTwoColumnTable provides a convenient method to include a table in your
% document with two tables.  
%
% #1 - path to data file
% #2 - column 1 heading
% #3 - column 2 heading
% #4 - section name
% #5 - section heading
% #6 - section text
% ------------------------------------------------------------------------------
\newcommand{\includeTwoColumnTable}[6]{%
  \redefineDataParameters{#1}{#2}{#3}{#4}{#5}{#6}%

  \IfFileExists{\dataFile}{%

    % Add a document section.
    \phantomsection%
    % \addcontentsline{toc}{section}{\dataSectionName}%
    \section{\dataSectionName}%
    \label{sec:\dataSectionTitle}%

    % Add section body.  Any well-formed LaTeX is allowed.
    \dataSectionBody%

    % Add a two column table. 
    \begin{center}%
      \csvreader[dataStyle, separator=pipe]%
        {\dataFile}% CSV file name
        {}% Filtering.
        {\thecsvrow&%
         \dataName&%
         \dataValue%
        }%
    \end{center}%
    \clearpage\pagebreak%
  }{% Else (Uncomment next line to debug files.)
    % \todo[inline]{\dataFile\ Missing input csv file.}%
  }% 
}%
% ------------------------------------------------------------------------------
