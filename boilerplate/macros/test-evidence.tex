% ------------------------------------------------------------------------------
% This set of commands define the path test-output/factory the amber driver
% produces as it process YAML files.
% ------------------------------------------------------------------------------
\newcommand{\tcCsv}{}%
\newcommand{\tcCsvCap}{\tcCsv}%
\newcommand{\tcCsvResult}{}%
\newcommand{\tcCsvResultCap}{\tcCsvResult}%
\newcommand{\tcLog}{}%
\newcommand{\tcPng}{}%
\newcommand{\tcPngCap}{}%

% define result counters
\newcounter{passcount}
\newcounter{failcount}
\newcounter{missingcount}
\newcounter{totalcount}

\newboolean{MoreSteps}\setboolean{MoreSteps}{true}%
\newcounter{StepCnt}\setcounter{StepCnt}{1}%

\newboolean{MoreFiles}\setboolean{MoreFiles}{true}%
\newcounter{FileCnt}\setcounter{FileCnt}{1}%

\newcommand{\tcCsvCapData}{%
  \begin{flushleft}%
    \scriptsize%
    \textbf{Step:~}\zerofill{\theStepCnt}~%
    \textbf{File:~}\zerofill{\theFileCnt}%
    \newline%
    \textbf{~Name:~}\tcCsv%
  \end{flushleft}%
}%

\newcommand{\tcCsvResultCapData}{%
  \begin{flushleft}%
    \scriptsize%
    \textbf{Step:~}N/A~%
    \textbf{File:~}N/A%
    \newline%
    \textbf{~Name:~}\tcCsvResult%
  \end{flushleft}%
}%

\newcommand{\tcPngCapData}{%
  \begin{flushleft}%
    \scriptsize%
    \textbf{Step:~}\zerofill{\theStepCnt}~%
    \textbf{File:~}\zerofill{\theFileCnt}%
    \newline%
    \textbf{~Name:~}\tcPng%
  \end{flushleft}%
}%

\newcommand{\setTOFPath}[1]{%
  \renewcommand{\tcCsv}{#1-\zerofill{\theStepCnt}-\zerofill{\theFileCnt}.csv}%
  \renewcommand{\tcCsvCap}{\tcCsvCapData}%
  \renewcommand{\tcCsvResult}{#1.csv}%
  \renewcommand{\tcCsvResultCap}{\tcCsvResultCapData}%
  \renewcommand{\tcLog}{#1.tex}%
  \renewcommand{\tcPng}{#1-\zerofill{\theStepCnt}-\zerofill{\theFileCnt}.png}
  \renewcommand{\tcPngCap}{\tcPngCapData}%
}%

\newcommand{\setTOFPathC}[3]{%
  \renewcommand{\tcCsv}{#3-\zerofill{\theStepCnt}-\zerofill{\theFileCnt}.csv}%
  \renewcommand{\tcCsvCap}{\tcCsvCapData}%
  \renewcommand{\tcCsvResult}{#3.csv}%
  \renewcommand{\tcCsvResultCap}{\tcCsvResultCapData}%
  \renewcommand{\tcLog}{#3.tex}%
  \renewcommand{\tcPng}{#3-\zerofill{\theStepCnt}-\zerofill{\theFileCnt}.png}
  \renewcommand{\tcPngCap}{\tcPngCapData}%
}%

% ------------------------------------------------------------------------------
% Comma Separated Variable table definition.  This table is applicable to 1 and
% 3 argument commands.
% ------------------------------------------------------------------------------
% Define column names use by csvreader for Test Case results.
\csvnames{tcNames}{1=\id, 2=\key, 3=\des, 4=\status}%

\csvstyle{tcStyle}{%
  longtable=|L{0.5cm}|L{1.6cm}|L{5.0cm}|L{6.0cm}|C{0.6cm}|
  ,table head=\hline\rowcolor{blue!35}\# & ReqId & ResKey & Text & Status\ER\endhead
  ,late after line=\ER
  ,table foot=\hline
  ,tcNames
}%

% ------------------------------------------------------------------------------
% Zero fill numbers
\newcommand{\zerofill}[1]{\ifnum#1<10 00#1\else\ifnum#1<100 0#1\else #1\fi\fi}%

% ------------------------------------------------------------------------------
% These commands are used when the amber driver is directed to run a Test Plan,
% Test Suite, or Test Case that does not provide support for a web browser or
% spoken language.
%
% Test Plan, Test Suite, and Test Case one argument commands:
% #1 - filename
% ------------------------------------------------------------------------------
% Define Test Plan Output (tpo) commands.
\newcommand{\tpo}[1]{\setTOFPath{#1}\inputIfExists{#1.tex}}%

% ------------------------------------------------------------------------------
% Define Test Suite Output (tso) commands.
\newcommand{\tso}[1]{\setTOFPath{#1}\inputIfExists{#1.tex}}%

% ------------------------------------------------------------------------------
% Define Test Case Output (tco) commands.
%
% Define a command to standardize test case output (tco).
\newcommand{\tco}[1]{\setTOFPath{#1}\assembleTestCaseOutput}%

% ------------------------------------------------------------------------------
% These commands are used when the amber driver is directed to run a Test Plan,
% Test Suite, or Test Case to support a web browser and spoken language.
%
% Test Plan, Test Suite, and Test Case three argument commands:
% #1 - browser
% #2 - language
% #3 - filename
% ------------------------------------------------------------------------------
% Define Test Plan Output (tpo) commands.
\newcommand{\tpoC}[3]{\setTOFPathC{#1}{#2}{#3}\inputIfExists{#3.tex}}%

% ------------------------------------------------------------------------------
% Define Test Suite Output (tso3) commands.
\newcommand{\tsoC}[3]{\setTOFPathC{#1}{#2}{#3}\inputIfExists{#3.tex}}%

% ------------------------------------------------------------------------------
% Define Test Case Output (tcoC) commands.
%
% Define a command to standardize test case output (tcoC).
\newcommand{\tcoC}[3]{\setTOFPathC{#1}{#2}{#3}\assembleTestCaseOutput}%

% ------------------------------------------------------------------------------
% Include the log file Amber created.
% ------------------------------------------------------------------------------
\newcommand{\includeLogFile}{\inputIfExists{\tcLog}}%

% ------------------------------------------------------------------------------
% Include the CSV file the program Amber called created.
% #1 - CSV file name. 
% #2 - CSV Caption 
% ------------------------------------------------------------------------------
\newcommand{\includeCsvFile}[2]{%
 \setSummaryCounters
  \IfFileExists{#1}{%
    #2% CSV Caption
    \begin{center}%
      \tiny%
      \csvreader[tcStyle, separator=pipe]%
        {#1}% CSV file name
        {}%   Filtering.
        {\thecsvrow\istotal\ispass\isfail\ismissing & \id & \key & \des & \status}%
    \end{center}%
     \clearpage\pagebreak%
  }{}%
}%

% ------------------------------------------------------------------------------
% Summary graph
% ------------------------------------------------------------------------------
% define counters

\newcommand{\setSummaryCounters}
{
  % set to zero
  \setcounter{passcount}{0}
  \setcounter{failcount}{0}
  \setcounter{missingcount}{0}
  \setcounter{totalcount}{0}
}

\newcommand{\ispass}{%
  \ifnum\pdfstrcmp{\status}{PASS}=0
    \stepcounter{passcount}%
  \fi
}

\newcommand{\isfail}{%
  \ifnum\pdfstrcmp{\status}{FAIL}=0
    \stepcounter{failcount}%
  \fi
}

\newcommand{\ismissing}{%
  \ifnum\pdfstrcmp{\status}{MISSING}=0
    \stepcounter{missingcount}%
  \fi
}

\newcommand{\istotal}{%
   \stepcounter{totalcount}%
}

% ------------------------------------------------------------------------------
% calculated percentage in case we want to show it

\newcommand{\percPASS}{%
   \the\numexpr(\thepasscount*100 / \thetotalcount)\relax%
}
\newcommand{\percFAIL}{%
   \the\numexpr(\thefailcount*100 / \thetotalcount)\relax%
}
\newcommand{\percDETECTED}{%
   \the\numexpr((\thepasscount + \thefailcount))*100 / \thetotalcount)\relax%
}

\newcommand{\percMISSING}{%
   \the\numexpr(\themissingcount*100 / \thetotalcount)\relax%
}

% ------------------------------------------------------------------------------
% calculate detected
\newcommand{\detectedcount}{%
  \the\numexpr(\thepasscount + \thefailcount)\relax%
}

% ------------------------------------------------------------------------------
% Include the summary chart
% ------------------------------------------------------------------------------
\newcommand{\includeSummaryChart}[1]
{
  \IfFileExists{#1}{%
   \begin{tikzpicture}
     \pie[pos={0,0}, explode=0.1, text=legend, radius=2, sum=auto]{\thepasscount/PASS,\thefailcount/FAIL}%
     \pie[pos={8,0}, explode=0.1, text=legend, radius=2, sum=auto, %
     color={green!\detectedcount,red!\themissingcount}]{\detectedcount/DETECTED, %
     \themissingcount/MISSING}%
    \end{tikzpicture}
    }{}%
}

% ------------------------------------------------------------------------------
% Include the PNG file the program Amber called created.
% ------------------------------------------------------------------------------
\newcommand{\includePngFile}{%
  \IfFileExists{\tcPng}{%
    \clearpage\pagebreak%
    \begin{figure}[ht]%
      \tcPngCap%
      \centering%
      \includegraphics[scale=0.31]{\tcPng}%
    \end{figure}%
    \includeCsvFile{\tcCsv}{\tcCsvCap}%
    \addtocounter{FileCnt}{1}%
  }%
  {% Else
    \setboolean{MoreFiles}{false}%
    \setcounter{FileCnt}{1}%
  }%
}

% ------------------------------------------------------------------------------
% Include test evidence which is a png file followed by a csv when present.
% ------------------------------------------------------------------------------
\newcommand{\includeTestEvidence}{%
  \setboolean{MoreSteps}{true}%
  \setcounter{StepCnt}{1}%
  %
  \setboolean{MoreFiles}{true}%
  \setcounter{FileCnt}{1}%
  %
  \whiledo{\boolean{MoreSteps}}{%
    \setboolean{MoreFiles}{true}%
    \IfFileExists{\tcPng}{%
      \whiledo{\boolean{MoreFiles}}{%
        \includePngFile%
      }%
      \addtocounter{StepCnt}{1}%
    }%
    {% Else
      \setboolean{MoreSteps}{false}%
      \setcounter{StepCnt}{1}%
    }%
  }%
}%

% ------------------------------------------------------------------------------
% Assemble test case output uses includeLogFile, includeTestEvidence, and 
% includeCsvFile.
% ------------------------------------------------------------------------------
\newcommand{\assembleTestCaseOutput}{%
  \includeLogFile%
  \includeTestEvidence%
  \includeCsvFile{\tcCsvResult}{\tcCsvResultCap}%
  \includeSummaryChart{\tcCsvResult}
 }%
% ------------------------------------------------------------------------------
