\subsection{General Terms}
\begin{description}

\item[Configuration Control] \quad The systematic process for managing changes
  to and established baseline.

\item[Configuration Identification] \quad A unique identifier used to associate
  a collection of software artifacts.

\item[Configuration Items] \quad Software source code, executables, build
  scripts, and other software development and software test artifacts relevant
  to creating and maintaining a software project.

\item[Configuration Status Accounting] \quad The recording and reporting of the
  information needed to effectively manage the software and documentation
  components of a software project.

\item[Report Package] \quad A detailed record that provides a Configuration
  Item overview, a list of its Intended Use Requirements (IUR), one or more Test
  Reports, evidence the Test Reports ran along with the output produced by
  Test Report including a pass/fail result for each Test Step and Test
  Report, and a statement indicating the Configuration Item has
  a Configuration Identification, and conclusion that the Configuration Item has
  been validated for its intended use.

\item[Test Plan] \quad A test plan is a collection of one or more test suites
  a tester has determined to use to challenge requirements.

\item[Test Suite] \quad A test suite is a collection of one or more test cases 
  a tester has determined to use to challenge requirements.

\item[Test Case] \quad A test case is a set of conditions under which a tester
  will determine whether the test is working as it was originally established
  for it to do.

\item[Test Step] \quad A unique test identifier with predetermined expectation,
  confirmation criteria, and pass/fail result.

\item[Test Report] \quad A test report consists of Detailed instructions for
  the set-up, execution, and evaluation of results for a given test.  The test
  protocol may include one or more test cases for which the steps of the
  protocol will repeat with different input data. Test cases are chosen to
  ensure that corner cases in the code and data structures are covered. A test
  protocol may be a script that is automatically run by the computer.
\end{description}
