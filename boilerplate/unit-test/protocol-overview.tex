\newpage
\section{Test Protocol Overview}
This section describes Test Protocols and Test Steps that demonstrate how
Configuration Item satisfies the IUR.  Each Test Protocol describes any setup
criteria needed to conduct the Test Steps. Each Test Protocol contains a list of
IUR and the Test Steps that demonstrate how the Configuration Item satisfies the
IUR. Each Test Step is marked passed or failed as it is completed.  Each Test
Protocol is marked passed when all Test Steps pass or failed if a single Test
Step fails.  This serves as a record of completed Test Protocols and Test Steps.

Each Test Protocol is described in its own section.  The order the Test
Protocols are listed is the order they are run.  Each Test Protocol defines:
\begin{description}

\item[Overview] \quad
Describes the intent and the approach.

\item[Test Steps] \quad
A unique test identifier with predetermined expectation,
confirmation criteria, and pass/fail result.

\item[Objective Evidence] \quad
A record the Test Protocol was run along with any evidence collected while
the Test Steps were run.

\item[Traceability] \quad
Test Steps are traced to IUR that is challenged. IUR can be traced to
multiple Test Protocols and Test Steps.

\end{description}

Test Steps are defined by a step number, a command to run, and arguments to the
command.  Each Test Protocol and Test Step has been designed to be run by the
computer.  However, a person may choose to manually run the Test Protocols,
save the test results, and generate this test report as specified in the
appropriate design documentation.

The example below is an automated script to that runs two commands: 1) git help
and 2) cat ~/gitconfig.  The first command is c="git" and its argument is
a=["help"].  The second command is c="cat" and its arguments is
a=[".gitconfig"].

\begin{verbatim}
For example:

  githelp = [(1, Step {r="IUR01", c="git", a=["help"]})
            ,(2, Step {r="IUR01", c="cat", a=[".gitconfig"]})
            ]

  Would be entered into an operating system shell console window:
        git help
        cat .gitconfig
\end{verbatim}

