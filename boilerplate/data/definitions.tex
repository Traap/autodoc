% {{{ Define glossary entries.
%
% \docTerm is defined in boilerplate/macros/utilities.tex.
%
% \docTerm uses three parameters to define a glossary entry:
%   #1 - key
%   #2 - name
%   #3 - description
%
% Notes:
%   1) Description can have blank lines and limited LaTeX syntax.
%
%   2) glossary entries do not need to be added alphabetically.  Although every
%      effort has been made to do so.
%
% \docTermPlural adds a fourth parameter:
%   #4 - plural form of #2.
%
% -------------------------------------------------------------------------- }}}
% {{{ [0..0]

\docTerm{510k}{510(k)}{%
  A premarket submission made to \Gls{fda} to demonstrate that a medical device to be
  marketed is at least as safe and effective, that is, substantially equivalent,
  to a legally marketed device (21 CFR 807.92(a)(3)) that is not subject to PMA.
}%

% -------------------------------------------------------------------------- }}}
% {{{ [A]

\docTerm{abbr510k}{abbreviated 510(k)}{%
  A type of alternative to the Traditional 510(k) pre-market notification that
  reduces the amount of information supplied to the \Gls{fda} in the pre-market
  notification and facilitates \Gls{fda} review utilizing:
  1) Guidance documents,
  2) Special controls, and
  3) Recognized standard.
}%

\docTerm{accept-number}{accept number}{%
  The maximum number of defective units in a single sampling plan using
  attribute data that will allow the lot to be accepted
}%

\docTerm{acceptance-criteria}{acceptance criteria}{%
  1) Specified set of operating parameter values that, if deviated from,
     renders the product quality in question
  2) Criteria a software product
     shall meet to successfully complete a test phase or to achieve delivery
     requirement.
}%

\docTermPlural{accessory}{accessory}{%
  An article, which while not being a device, is intended specifically by its
  manufacturer to be used together with a device to enable it to be used in
  accordance with the manufacturer's intended use of the device.
}{accessories}%

\docTerm{action-plan}{action plan}{%
  Plan to address root cause via Corrective or Preventive Action.
}%

\docTerm{active-record}{active record}{%
  A record that has sufficient value to warrant its use for current operations
  and should remain available for immediate retrieval.
}%

\docTerm{adverse-event}{adverse event}{%
  Any undesirable experience associated with the use of a medical product in
  a patient. The event is serious and should be reported to \Gls{fda} when the patient
  outcome results in death, life-threatening, hospitalization, disability,
  permanent damage, congenital anomaly / birth defect or required intervention.
}%

\docTerm{advisory-notice}{advisory notice}{%
  Notice issued by the organization, subsequent to delivery of the medical
  device, to provide supplementary information and/or to advise what action
  should be taken in the use of a medical device, the modification of a medical
  device, the return to the organization that supplied the medical device, or
  the destruction of a medical device, for the purpose of corrective or%
  preventive action and in compliance with national and regional regulatory
  requirements.
}%

\docTerm{affected-functions}{affected functions}{%%
  Departments or functional areas that are affected by the document or process;
  should be included in the document's approval process; a department head or
  a designee represents the affected functions.
}%

\docTerm{annual-report}{annual report}{%
  Report submitted annually to the \Gls{fda} by the applicant of a New Drug
  Application (NDA) or an Abbreviated New Drug Application (ANDA) as required
  per 21 CFR 314.81(b)(2).
}%

\docTerm{amber}{amber}{%
  A \gls{rubygem} that is used for automated testing.
}%

\docTermPlural{anomaly}{anomaly}{%
  (IEEE) Anything observed in the documentation or operation of software that
  deviates from expectations based on previously verified software products
  or reference documents.
}{anomalies}%

\docTerm{approval}{approval}{%
  The signature of the individuals(s) approving the entry, entries, or
  document(s), and the date of the signature, whether by paper or electronic
  means.
}%

\docTerm{approved-supplier}{approved supplier}{%
  Suppliers that have demonstrated evidence of being able to meet \gls{company}%
  requirements.
}%

\docTerm{architecture}{architecture}{%
  The framework of physical system components and structural properties that
  establish the basis for system implementation and operation.
}%

\docTerm{audit}{audit}{%
  A systematic and independent examination to determine whether:
  1) Quality activities and related results comply with requirements.
  2) These requirements are implemented effectively.
  3) The requirements are suitable to achieve end product or service quality.
  4) Objective evidence exists demonstrating the above.
}%

\docTerm{audit-criteria}{audit criteria}{%
  Sets of policies, procedures or requirements, including regulations,
  standards, and directives, used as a reference against which Audit evidence is
  compared.
}%

\docTerm{audit-focus}{audit focus}{%
  Identification of individual topics and associated regulation and standard
  elements audited with the intent of demonstrating coverage of regulations and
  standards.
}%

\docTerm{audit-objective}{audit objective}{%
  What will be accomplished during the audit, including determination of the
  extent of conformity, evaluation of capability of a system requirements and
  identification of potential improvements.
}%

\docTerm{audit-plan}{audit plan}{%
  A document that describes the objectives, audit scope, dates, audit criteria
  and audit focus for a specific audit.
}%

\docTerm{audit-scope}{audit scope}{%
  The identified organization(s), functional areas, activities and/ or processes
  to be audited make up the audit scope.
}%

\docTerm{auditee}{auditee}{%
  Organization or system being audited.
}%

\docTerm{auditor}{auditor}{%
  Person or team of individuals with competence to conduct a quality audit.
}%

\docTerm{awaredate}{aware date}{%
  The date Acme Corporation has determined a complaint to meet requirements for reporting
  to the applicable regulatory body.  This date is commonly the notification
  date, unless:
  1) Not enough information was acquired with the initial report to make the
     determination; or
  2) Additional information is acquired that results in a decision to change
     a prior determination of Non-reportable to Reportable.
}%

\docTerm{autodoc}{autodoc}{%
  A dynamic and static \LaTeX\ documentation provider.
}%

% -------------------------------------------------------------------------- }}}
% {{{ [B]

\docTerm{batch}{batch}{%
  One or more components or finished devices that consist of a single type,
  model, class, size, composition or software version that are manufactured
  under essentially the same conditions and that are intended to have uniform
  characteristics and quality within specified limits. A Batch can be made up of
  sub-assemblies (SA's), Finished Components (FC's) or Packaging Finished
  Components (EI's).
}%

\docTerm{batch-number}{batch number}{%
  A unique identifier assigned to a particular batch/lot for traceability and
  identification purposes.
}%

\docTerm{beta-evaluation}{beta evaluation}{%
  Installation and operation of computer software at customer or other non
  \gls{company} locations following Validation of the software and prior to
  Production Release of the software, for the purpose of evaluating the
  software's operation and performance.  This may be performed in conjunction
  with Market Evaluation, Clinical Trial, Field Trial, or other arrangements as
  allowed by \gls{company} process.
}%

\docTerm{biohazard}{biohazard}{%
  Biological agent or substance present in or arising from the work environment
  that present or may present a hazard to the health or well-being of the worker
  or community.
}%

\docTerm{build}{build}{%
  The process of generating a product's Software Configuration Item
  (deliverables) from its Software Configuration Components (sources).
}%

\docTerm{business-requirement}{business requirement}{%
  Requirements established by \gls{company} that do not fulfill Regulatory or
  Quality requirements, but assure consistent performance of other business
  activities.
}%

\docTerm{business-unit}{business unit}{%
  A \gls{company} organizational entity identified to focus on unique products
  or disease groups to meet the needs of specific patient and physician
  populations.
}%

\docTerm{backdating}{backdating}{%
  backdating, or post dating occurs when the signature and date do not reflect
  the date the information was entered.
}%

\docTerm{bamboo}{Bamboo}{%
  An Atlassian Bamboo Server.
}%

\docTerm{baseline}{baseline}{%
  A software version that constitutes a basis for further development, has been
  assigned a unique configuration identification, and can be changed only
  through \gls{configuration-control} A baseline consists of the initial baseline
  plus any approved changes to that baseline.
}%

\docTerm{bitbucket}{Bitbucket}{%
  An Atlassian Bitbucket Server.
}%

\docTerm{bundler}{bundler}{%
  A \gls{rubygem} package manager.
}%

% -------------------------------------------------------------------------- }}}
% {{{ [C]

\docTerm{calibration}{calibration}{%
  Process for ensuring continuous adequate performance of sensing, measurement,
  and actuating equipment with regard to specified accuracy and precision
  requirements.
}%

\docTerm{capabilityanalysis}{capability analysis}{%
  A statistical measure of ability to meet an established performance
  specification, typically expressed as Cpk or Ppk.
}%

\docTerm{cause}{cause}{%
  The conditions which may cause the Hazard to occur.
}%

\docTerm{cemark}{ce mark}{%
  Approval to market medical devices in European Countries.
}%

\docTerm{certification}{certification}{%
  The demonstration, by objective evidence of process controls, that results in
  the reduction or elimination of incoming inspection of a Product.
}%

\docTerm{changecontrol}{change control}{%
  The processes, authorities for, and procedures to be used for all changes that
  are made to the computerized system and/or the system's data.  Change
  control is a vital subset of the Quality Assurance (QA) program within an
  establishment and should be clearly described in the establishment's SOP.
}%

\docTerm{client-server}{client-server}{%
  (FDA) A term used in a broad sense to describe the relationship between the
  receiver and the provider of a service. In the world of microcomputers, the
  term client-server describes a networked system where front-end
  applications,                                                           as the client,    make service requests upon another networked
  system. Client-server relationships are defined primarily by software. In a
  local area network (LAN),                                               the workstation is the client and the file server
  is the server. However,                                                 client-server systems are inherently more complex
  than file server systems. Two disparate programs must work in tandem,   and
  there are many more decisions to make about separating data and processing
  between the client workstations and the database server. The database
  server encapsulates database files and indexes,                         restricts access, enforces
  security,                                                               and provides applications with a consistent interface to data via
  a data dictionary.
}%

\docTerm{clinical-evaluations}{clinical evaluations}{%
  Studies of an investigational and/or marketed medical device and/or drug
  involving human subjects conducted to assess the function, safety, or efficacy
  of the device/drug including evaluations for regulatory submission.
}%

\docTerm{clinical-protocol}{clinical protocol}{%
  A detailed plan for the performance of a clinical research study, which does
  not involve exposure of donors to experimental articles.  The plan will:
  1) Identify the objectives of the study.
  2) Provide brief background information, instructions, and details of the
     steps to be followed in conducting the study.
}%

\docTerm{clinical-trial}{clinical trial}{%
  A systematic collection of data, involving one or more human subjects, for the
  purpose of determining the safety, efficacy, and/or function of an
  investigational or marketed drug, biologic, device or of an investigational,
  or approved medical intervention.  This excludes the use of a marketed drug in
  the course of medical practice.
}%

\docTerm{clinical-study}{clinical study}{%
  see Clinical Trial.
}%

\docTerm{clinical-investigation}{clinical investigation}{%
  see Clinical Trial.
}%

\docTerm{combination-product}{combination product}{%
  The term combination product includes:
  1) A product comprised of two or more regulated components, i.e., drug/device,
     biologic/device, drug/biologic, or drug/device/biologic, that are
     physically, chemically, or otherwise combined or mixed and produced as
     a single entity.
  2) Two or more separate products packaged together in a single package or as
     a unit and comprised of drug and device products, device and biological
     products, or biological and drug products.
  3) A drug, device, or biological product packaged separately that according to
     its investigational plan or proposed labeling is intended for use only with
     an approved individually specified drug, device, or biological product
     where both are required to achieve the intended use, indication, or effect
     and where upon approval of the proposed product the labeling of the
     approved product would need to be changed, e.g., to reflect a change in
     intended use, dosage form, strength, route of administration, or
     significant change in dose; or
  4) Any investigational drug, device, or biological product packaged separately
     that according to its proposed labeling is for use only with another
     individually specified investigational drug, device, or biological product
     where both are required to achieve the intended use, indication, or effect.
}%

\docTerm{commodity}{commodity}{%
  A term used to describe categories of purchased Products or Services. E.g.
  Plastics, Metals, Chemicals.
}%

\docTerm{compliant}{compliant}{%
  820.3(b) means any written, electronic, or oral communications that alleges
  deficiencies related to the identity, quality, durability, reliability,
  safety, effectiveness, or performance of a device after it is released to
  distribution.
}%

\docTerm{computer}{computer}{%
  (IEEE) (1) A functional unit that can perform substantial computations,
  including numerous arithmetic operations, or logic operations, without
  human intervention during a run. (2) A functional programmable unit that
  consists of one or more associated processing units and peripheral
  equipment, that is controlled by internally stored programs, and that can
  perform substantial computations, including numerous arithmetic operations,
  or logic operations, without human intervention.
}%
\docTerm{computer-system}{computer system}{%
  (ANSI) a functional unit, consisting of one or more computers and
  associated peripheral input and output devices, and associated software,
  that uses common storage for all or part of a program and also for all or
  part of the data necessary for the execution of the program; executes
  user-written or user-designated programs; performs user-designated data
  manipulation, including arithmetic operations and logic operations; and
  that can execute programs that modify themselves during their execution. A
  computer system may be a stand- alone unit or may consist of several
  interconnected units.  See: computer, computerized system.
}%

\docTerm{computerized-system}{computerized system}{%
  Includes hardware, software, peripheral devices, personnel, and
  documentation; e.g., manuals and Standard Operating Procedures.
}%

\docTerm{component}{component}{%
  820.3(c) means any raw material, substance, piece part, software, firmware,
  labeling, or assembly which is intended to be included as part of the
  finished, packaged, and labeled device.
}%

\docTerm{configurable-software}{configurable software}{%
  Commercially available software which provides the user with the capability
  for defining operating parameters, data collection/analysis procedures,
  limits, reports, and other operational controls.
}%

\docTerm{configuration}{configuration}{%
  (IEEE)
  1) The arrangement of a computer system or component as defined by the number,
     nature, and interconnections of its constituent parts;
  2) In configuration management, the functional and physical characteristics of
     hardware or software as set forth in technical documentation or achieved in
     a product.
}%

\docTerm{configuration-control}{configuration control}{%
  The systematic process for managing changes to an establish baseline.
}%

\docTerm{configuration-identification}{configuration identification}{%
  A unique identifier used to associate a collection of software artifacts. The
  form of major.minor.patch.build is used where each part represents a number
  such as 1.0.0.0, where
  \textbf{major} means incompatible changes occur, and or new
    features are added;
  \textbf{minor} means backwards-compatible features are
    added;
  \textbf{patch} means backwards-compatible bug fixes are applied; and
  \textbf{build} is incremented for every build.
  Version Identification is used to define each Development, Integration, and
  Production release candidate.
}%

\docTerm{configuration-item}{configuration item}{%
  Software source code, executables, build scripts, and other software
  development and software test artifacts relevant to creating and maintaining
  a software product.
}%

\docTerm{configuration-management}{configuration management}{%
  (IEEE)
  A discipline applying technical and administrative direction and surveillance
  to identify and document the functional and physical characteristics of
  a configuration item, control changes to those characteristics, record and
  report change processing and implementation status, and verifying compliance
  with specified requirements.
}%

\docTerm{configuration-status-accounting}{configuration status accounting}{%
  The recording and reporting of the information needed to effectively manage
  the software and documentation components of a software project.
}%

\docTerm{consignee}{consignee}{%
  1) The site that receives product;
  2) The person or company to whom imported Goods is shipped or who has
     authority to be designated importer of record on customs entry
     documents.
}%

\docTerm{consignor}{consignor}{%
  The person or company shown on the bill of lading as the shipper.
}%

\docTerm{consultant}{consultant}{%
  A person or group or people who provide specialized advice or expertise to
  \gls{company}.
}%

\docTerm{containment}{containment}{%
  Steps taken to prevent the release or spread of a potential or existing
  nonconformity.
}%

\docTerm{content}{content}{%
  All information contained within the quality system (documents, data, etc.)
  that establishes regulatory, quality and business process requirements and
  enables the operation of the system.
}%

\docTerm{content-engineering}{content re-engineering}{%
  Evaluating and revising content to assure consistency, reduce redundancy,
  address requirement gaps, optimize architecture, and increase usability.
}%

\docTerm{contract-product-developer-or-manufacturer}{contract product developer
  or manufacturer}{%
  An external supplier that designs or builds finished devices distributed by
  \gls{company}, where \gls{company} maintains control/approval of the device
  specifications.
}%

\docTerm{control}{control}{%
  Requirements for control of Quality Management System documents include:
  1) Review and approval of documents for adequacy prior to issue;
  2) Review and update as needed and re-approval as necessary;
  3) Changes to documents and current revision status is identified;
  4) Relevant versions are available at points of use; and
  5) Documents are legible and identifiable
}%

\docTerm{control-number}{control number}{%
  820.3(d) means any distinctive symbol, such as a distinctive combination of
  letters numbers, or both, from which the history of the manufacturing,
  packaging, labeling, and distribution of a unit, lot, or batch of finished
  devices can be determined.
}%

\docTerm{control-plan}{control plan}{%
  A document that identifies key manufacturing or service process steps or
  inputs and how those items will be sustained.
}%

\docTerm{controlled-document}{controlled document}{%
  A document that requires review, approval, and issuance through a defined and
  documented process.
}%

\docTerm{corporate}{corporate}{%
  A \gls{company} organizational entity identified to establish and oversee the
  \gls{company} business and all global \gls{company} requirements.
}%

\docTerm{corporate-global-document}{corporate (global) document}{%
  Consist of \gls{company} Corporate (Bad Homburg, Germany) policies,
  procedures, work instructions, forms, guidance and any associated training
  documents as they apply to Quality, Regulatory and Clinical Research
  activities.
}%

\docTerm{correction}{correction}{%
  Action taken to eliminate a detected nonconformity. (CAPA context). Repair,
  modification, adjustment, relabeling, destruction, or inspection (including
  patient monitoring) of a product without its physical removal to some other
  location. (Field Action context).
}%

\docTerm{corrective-action}{corrective action}{%
  Action taken to eliminate the cause(s) and prevent recurrence of an existing
  product, process, or quality system non-conformance, defect, trend  or or other
  undesirable situation.
}%

\docTerm{critical-control-point}{critical control point}{%
  A decision, calculation or communications interface at which control can be
  applied and, as a result, a hazard can be prevented, eliminated, or reduced to
  an acceptable level.
}%

\docTerm{critical-defect}{critical defect}{%
  Defect that could cause clinical harm.
}%

\docTerm{cyber-security}{cyber security}{%
  The process of preventing unauthorized modification, misuse or denial of use,
  or the unauthorized use of information that is stored, accessed, or transferred
  from a medical device to an external recipient.
}%

% -------------------------------------------------------------------------- }}}
% {{{ [D]

\docTerm{defect}{defect}{%
  A departure of a single quality characteristic of a product or service that
  deviates from the desired requirements or specifications.  One (1) defect may
  render a unit defective.
}%

\docTerm{defective}{defective}{%
  A unit of product or service containing at least one (1) defect that causes
  the unit to not satisfy the intended requirements.
}%

\docTerm{design-change}{design change}{%
  All changes to the device or manufacturing process design, including those
  occurring long after a device has been introduced to the market. This includes
  evolutionary changes, such as performance enhancements as well as
  revolutionary changes, such as corrective actions resulting from the analysis
  of failed product. The changes are part of a continuous, ongoing effort to
  design and develop a device that meets the needs of the user and/or patient.
}%

\docTerm{design-dossier}{design dossier}{%
  For products certified through Annex III or Annex II.4, the Technical File is
  called Design Dossier and is submitted to the Notified Body for conformity
  assessment when applying for CE Mark.
}%

\docTerm{design-transfer}{design transfer}{%
  Design Transfer is the process for translating design requirements into
  production specifications to ensure consistent and acceptable production
  results.
}%

\docTerm{design-verification}{design verification}{%
  Confirmation by objective evidence that design output fulfills the design
  input requirements.
}%

\docTerm{destructive-analysis}{destructive analysis}{%
  The physical modification or disassembly of the returned product, excluding
  decontamination and sterilization, and in the case of electronic devices, the
  modification of device settings or memory outside those required to turn
  ventricular fibrillation detection therapy off or to interrogate the
  device using a \gls{company} programmer.
}%

\docTerm{detectability}{detectability}{%
  Rate the likelihood that the potential failure will not be detected prior to
  intended user (primarily patient, sometimes clinician). Detection by the
  hospital, but after shipment and before patient use, is considered not
  detected.
}%

\docTerm{development-lifecycle}{development lifecycle}{%
  The period of time that begins when a product is conceived and ends when the
  product is no longer available for use. For purposes of this procedure, the
  development lifecycle is defined as requirements, design, implementation,
  system test, and operation and maintenance.
}%

\docTerm{deviation}{deviation}{%
  A detected or potential non-conformance, system deficiency, or non-fulfillment
  of specified requirements.  Can be product, process, or system related.
}%

\docTerm{device-identifier}{a device identifier}{%
  820.3(cc)(1) a mandatory, fixed portion of a UID that identifies the
  specific version or model of a device and the labeler of that device.
}%

\docTerm{direct-supplier}{direct supplier}{%
  An external supplier that provides a commodity that is utilized in/ as an ABT
  finished product.
}%

\docTerm{distributor}{distributor}{%
  A company that furthers the marketing and/or supply of product.
}%

\docTerm{docbld}{docbld}{%
  A dynamic and static \LaTeX\ documentation assembler.
}%

\docTerm{document-category}{document category}{%
  An identifier that establishes unique purpose and scope for a QMS Document.
  Document Categories include Policy, Top Level Procedure, Standard Operating
  Procedure, Work Instruction, Form and Supporting Material.
}%


\docTerm{document-owner}{document owner}{%
  Departmental management or designee assigned primary approval responsibility
  for a specific Quality System document; Ensures implementation and that all
  affected tiers are reviewed annually (unless otherwise noted) and there is
  evidence of this review.
}%

\docTerm{document-structure}{document structure}{%
  A component of QMS Architecture that establishes Document Categories and the
  Document Category hierarchy.
}%

\docTerm{drug-product}{drug product}{%
  A finished dosage form, for example, tablet, capsule, solution, etc., that
  contains an active drug ingredient generally, but not necessarily, in
  association with inactive ingredients. The term also includes a finished
  dosage form that does not contain an active ingredient but is intended to be
  used as a placebo.
}%

% -------------------------------------------------------------------------- }}}
% {{{ [E]

\docTerm{effectiveness-check}{effectiveness check / verification}{%
  Monitoring or assessments completed after implementation of Corrective and/or
  Preventive Actions to establish whether actions taken were effective and
  addressed the identified nonconformity.
}%

\docTerm{electronic-record}{electronic record}{%
  1) Any combination of text, graphics, data, audio, pictorial, or other
     information representation in digital form that is created, modified,
     maintained, archived, retrieved, or distributed by media.
  2) (FDA) Any combination of text, graphics, data, audio, pictorial, or other
     information representation in digital form that is created, modified,
     maintained, archived, retrieved, or distributed by a computer system.
}%

\docTerm{enabling-process}{enabling process}{%
  QMS Process that controls or supports Product Lifecycle Processes to ensure
  the suitability, adequacy and effectiveness of the QMS.
}%

\docTerm{enforcement-action}{enforcement action}{%
  Communication received from the State or Federal Food and Drug Administration
  including:  Warning Letter, Consent Decree and titled letters.
}%

\docTerm{essential-documents}{essential documents}{%
  Documents which:
  1) Individually and collectively permit evaluation of the conduct of a study
     and the quality of the data produced.
  2) Serve to demonstrate the compliance of the investigator, sponsor, and
     monitor with the standards of Good Clinical Practice and with all
     applicable regulatory requirement.
}%

\docTerm{established}{established}{%
  820.3(k) means defined, documented (in writing or electronically), and
  implemented.
}%

\docTerm{evaluation}{evaluation}{%
  The documented result of an analysis that demonstrates ability to meet
  specification requirements.
}%

\docTerm{excipient}{excipient}{%
  Those components of a finished medicinal drug product other than the API\@.
  They are included in the formulation to facilitate manufacture, enhance
  stability, control release of API from the product, assist in product
  identification or enhance other potential characteristics.
}%

\docTerm{executive-responsibility}{management with executive responsibility}{%
  Senior employees of a manufacturer who have the authority to establish or make
  changes to the manufacturer's quality policy and quality system.
}%

\docTerm{external-investigator}{external investigator}{%
  Investigator authorized by the agency performing an inspection (United States
  government, non U.S. government agency, or Notified Body).  Also, referred to
  as an External Auditor.
}%

\docTerm{external-requirements}{external requirements}{%
  Requirements that originate outside of the organizational entity.  External
  Requirements include Customer Requirements, Regulations, Standards and
  \gls{company} Corporate Policies.
}%

% -------------------------------------------------------------------------- }}}
% {{{ [F]

\docTerm{feature}{feature}{%
  A feature describes a part and/or its performance. Features include any
  dimensional, visual, functional, mechanical, electrical, chemical, physical,
  or material properties of a Product or Service.
}%

\docTerm{finished-device}{finished device}{%
  820.3(l) means any device or accessory to any device that is suitable for use
  or capable of functioning whether or not it is packaged, labeled or
  sterilized.
}%

\docTerm{finished-goods}{finished goods}{%
  Items manufactured by an outside supplier that require no further processing
  prior to being sold by Acme Corporation.
}%

\docTerm{firmware}{firmware}{%
  Combination of persistent memory and program code and data stored in it.
  Typical examples of devices containing firmware are embedded systems (such as
  process controllers, lab equipment), computers, computer peripherals, and
  mobile devices.
}%

\docTerm{fiscal-year}{fiscal year}{%
  An accounting period containing 12 continuous months.  For \gls{company}%
  a fiscal year generally runs from January 1st -- December 31st.
}%

\docTerm{form}{form}{%
  Document that is used as a method and/or tool to execute business processes.
  Forms, templates, and checklists are used to record required data, usually
  found in instructions, and when completed, become a record.
}%

\docTerm{functional-analysis}{functional analysis}{%
  (IEEE) Verifies that each safety-critical software requirement is covered
  and that an appropriate criticality level is assigned to each software
  element.
}%

% -------------------------------------------------------------------------- }}}
% {{{ [G]

\docTerm{git-account}{git account}{%
  A valid Git username and password.  Request these from your Git administrator.
}%

\docTerm{global-document}{global document}{%
  A document that applies to more than one site.
}%

\docTerm{global-it}{global information technology}{%
  The \gls{company} Information Technology organization consisting of all site
  and corporate organization reporting to the Chief Information Officer (CIO)
}%

\docTerm{governance}{governance}{%
  Organizational oversight that assures the development and management of
  consistent, effective process by defining expectations, granting power, and
  establishing criteria for decisions and performance.
}%

\docTerm{governancelevel}{governance level}{%
  An attribute associated with QMS Documents that identifies the organizational
  entity that controls the document.  This attribute does not establish document
  scope.  The Applicable Site attribute establishes document scope and may not
  be equal to the level of organizational oversight (governance level).
}%

\docTerm{group}{group}{%
  A \gls{company} organizational entity encompassing multiple Business Units,
  identified to improve operating effectiveness and efficiency through
  standardization of process and leverage of common resources.
}%

\docTerm{gxp}{gxp}{%
  General term for Good Practice quality guidelines and regulations.  E.g. Good
  Manufacturing Practices, Good Clinical Practices.
}%
% -------------------------------------------------------------------------- }}}
% {{{ [H]

\docTerm{harm}{harm}{%
  Physical injury or damage to the health of people, or damage to property or
  the environment.
}%

\docTerm{harmonized-standard}{harmonized standard}{%
  A harmonised standard is a European standard developed by a recognised
  European Standards Organisation: CEN, CENELEC, or ETSI\@. It is created
  following a request from the European Commission to one of these organisations.
}%

\docTerm{hazard}{hazard}{%
  The potential source of harm.
}%

\docTerm{hazardous-situation}{hazardous situation}{%
  Circumstance in which people, property, or the environment are exposed to one
  or more hazard(s).
}%

% -------------------------------------------------------------------------- }}}
% {{{ [I]

\docTerm{identifier}{identifier}{%
  A unique series of letters or numbers or any combination of these or a bar
  code that is assigned to a medical device by the manufacturer and that
  identifies it and distinguishes it from similar devices.
}%

\docTerm{illness}{illness}{%
  see serious injury.  An injury or illness that is life-threatening, even if
  temporary; results in permanent impairment of a body function or permanent
  damage to body structure; or necessitates medical or surgical intervention to
  preclude permanent impairment of a body function or permanent damage to a body
  structure.
}%

\docTerm{impact-assessment}{impact assessment/ review}{%
  A documented assessment of a proposed Change by Responsible Functions.  This
  assessment may include evaluation of the potential effects and consequences of
  a Change on product safety and efficacy, relevancy of product testing,
  validations, and regulatory geographic registrations.
}%

\docTerm{improvement-opportunity}{improvement opportunity}{%
  An area of the quality system identified as needing to perform at a higher
  level to achieve long-term organizational success.
}%


\docTerm{independent-reviewer}{independent reviewer}{%
  An individual who does not have direct responsibility for the items being
  reviewed but is knowledgeable in the subject area. This person acts as an
  objective reviewer and brings a fresh perspective to the team.
}%

\docTerm{indirect-supplier}{indirect supplier}{%
  External supplier that provides as service to \gls{company} or \gls{company}%
  customers.
}%

\docTerm{information-system}{information system}{%
  Software or a combination of software, firmware and hardware, with associated
  data, people, processes and documentation, integrated to accomplish a specific
  function or set of functions.
}%

\docTerm{inspection}{inspection}{%
  A detailed review of the Organization's compliance with applicable laws,
  regulations, standards and internal processes.  Also referred to as an
  external audit.  Inspections are performed by \Gls{fda}, Notified Bodies, PMDA, TGA,
  etc.
}%

\docTerm{installer}{installer}{%
  Installer is a software-only utility that is used to install and configure
  software into an operating system environment.
}%

\docTerm{intended-use}{intended use}{%
  Intended purpose or use for which a product, process or service is intended
  according to the specifications, instructions and information provided by the
  manufacturer.
}%

\docTerm{internal-requirements}{internal requirements}{%
  Requirements that originate within organizational entity.  Internal
  Requirements include Quality and Business Requirements, but not Regulatory
  Requirements.
}%

\docTerm{internal-supplier}{internal supplier}{%
  Suppliers that are other \gls{company} facilities/entities.
}%

\docTerm{investigation}{investigation}{%
  Review of CAPA PR/Event to determine the scope, depth and breadth of
  a non-conformance in order to determine Root Cause.
}%

\docTerm{investigational-device}{investigational device}{%
  A medical device or product that is being evaluated in a clinical study and is
  not approved / cleared or is different from the approved / cleared form or is
  being used for an unapproved or unclear indication/use.
}%

% -------------------------------------------------------------------------- }}}
% {{{ [J]

\docTerm{jira}{JIRA}{%
  An Atlassian project and issue-tracking platform.
}%

% -------------------------------------------------------------------------- }}}
% {{{ [L]

\docTerm{label}{label}{%
  Any written, printed, or graphic matter affixed to or appearing upon any
  consumer commodity or affixed to or appearing upon a package containing any
  consumer commodity.
}%

\docTerm{labeling}{labeling (noun)}{%
  All written, printed or graphic matter affixed to a medical device or any of
  its containers or wrappers, or accompanying a medical device, related to
  identification, technical description and use of the medical device, but
  excluding shipping documents.
}%

\docTerm{laboratory-notebook}{laboratory notebook}{%
  A bound, pre-numbered book with numbered pages and an index (or Table of
  Contents), which is the confidential property of the company and includes
  facility name and location. Laboratory notebooks provide a means of
  permanently recording daily work conducted in:
  1) Research,
  2) Development, and
  3) Support activities.
}%

\docTerm{latex}{\LaTeX}{%
  is a high-quality typesetting system, it includes features designed for the
  production of technical and scientific documents, and documents requiring
  precise typesetting requirements.
}%

\docTerm{legal-manufacturer}{legal manufacturer}{%
  Entity responsible for the design, manufacture, packaging and labelling of
  a device before it is placed on the market.
}%

\docTerm{life-of-product}{life of product}{%
  End of production plus 15 years or Life of Medical Device (LMD) plus 1 year,
  whichever is longer.
}%

\docTerm{lot}{lot (or batch)}{%
  One or more components or finished devices that consist of a single type,
  model, class, size, composition or software version manufactured under
  essentially the same conditions and intended to have uniform characteristics
  and quality within specified limits.
}%

\docTerm{lot-acceptance}{lot acceptance}{%
  Activity (through testing, inspection, or other verification methods) to
  ensure that purchased or otherwise received product or services conform to
  specifications and are acceptable for their intended use.
}%

% -------------------------------------------------------------------------- }}}
% {{{ [M]

\docTerm{maintenance}{maintenance}{%
  Activities such as adjusting, cleaning, modifying and overhauling equipment to
  assure performance in accordance with requirements. Maintenance of a software
  system includes correcting software errors, adapting software to a new
  environment or making enhancements to software.
}%

\docTerm{malfunction}{malfunction}{%
  A failure of the device to meet its performance specifications or otherwise to
  perform as intended. Performance specifications include all claims made in the
  labeling for the device, and include uses commonly applied in the medical
  community.  Use errors may also be considered to be a malfunction
  despite the fact that the device did not fail to meet specifications.  Note:
  A failure of a device to meet a specification that does not affect the
  performance (either intended or labeling claims) is not considered
  a malfunction, i.e., cosmetic defects.
}%

\docTerm{management}{management}{%
  820.3(n) with executive responsibility means those senior employees or
  a manufacturer who have the authority to establish or make changes to the
  manufacturer's quality policy and quality system.
}%

\docTerm{management-representative}{management representative}{%
  Management employees, appointed by Management with Executive Responsibility,
  who ensure QMS requirements are established and maintained, and report on
  performance of QMS to Management with Executive Responsibility.
}%

\docTerm{manufacturer}{manufacturer}{%
  820.3(o) means any person who designs, manufactures, fabricates, assembles, or
  process a finished device.  Manufacturer includes but is not limited to those
  who perform the functions of contract sterilization, installation, relabeling,
  re-manufacturing, repackaging, or specification development, and initial
  distributors of foreign entities performing these functions.
}%

\docTerm{manufacturing-escape}{manufacturing escape}{%
  Products that have been final released by the producing Manufacturing site and
  subsequently discovered to be non-conforming.
}%

\docTerm{manufacturing-material}{manufacturing material}{%
  Any material or substance used in or used to facilitate the manufacturing
  process, a concomitant constituent or a byproduct constituent produced during
  the manufacturing process, which is present in or on the finished device as
  a residue or impurity not by design or intent of the manufacturer.
}%

\docTerm{market-withdrawal}{market withdrawal}{%
  A correction or removal of a distributed device that involves a minor
  violation of the act that would not be subject to legal action by \Gls{fda} or that
  involves no violation of the act, e.g.\ normal stock rotation practices.
}%

\docTerm{medical-device}{medical device}{%
  An instrument, apparatus, implement, machine, contrivance, implant, in vitro
  reagent, or other similar or related article, including a component part or
  accessory which is recognized in the official National Formulary, of the
  United States Pharmacopoeia, or any supplement to them; intended for use in
  the diagnosis of disease or other conditions, or in the cure, mitigation,
  treatment or prevention of disease, in man or other animals; or  intended to
  affect the structure or any function of the body of man or other animals, and
  which does not achieve any of its primary intended purposes through chemical
  action within or on the body of man or other animals and which is not
  dependent upon being metabolized for the achievement of any of its primary
  intended purposes.
}%

\docTerm{medical-personnel}{medical personnel (medical director)}{%
  An individual who is licensed, registered or certified by a State, territory
  or other governing body to administer health care; has received a diploma or
  a degree in a professional or scientific discipline; is an employee
  responsible for receiving medical complaints or adverse event reports; or is
  a supervisor of such persons.
}%

\docTerm{mitigation}{mitigation}{%
  Steps taken to alleviate or lessen an issue, including Corrections and
  Containment; these steps are typically required when an issue is first
  identified but may also be required throughout the life of a CAPA..
}%

% -------------------------------------------------------------------------- }}}
% {{{ [N]

\docTerm{nearmiss}{near miss}{%
  Post completion of routine in-process inspections and testing detection
  of an issue that could potentially compromise patient safety if the material
  was released for customer use.  To qualify as a near miss there must be no
  subsequent assurance of detecting the issue.
}%

\docTerm{network-access}{network access}{%
  Your computer requires HTTP and HTTPS access to the Internet.  Contact your
  network administrator.
}%

\docTerm{nonconformity}{nonconformity}{%
  820.3(q) means the non-fulfillment of a specified requirement.
}%
\docTerm{non-routine-servicing}{non-routine servicing}{%
  Repairs of an unexpected nature, replacement of parts earlier than their
  normal life expectancy, or identical repairs or replacement of multiple units
  of a device are not routine servicing.
}%

\docTerm{notified.bodies}{notified bodies}{%
  Organizations recognized by a competent authority in the Member State to carry
  out CE (European Conformity) marking compliance procedures.
}%

% -------------------------------------------------------------------------- }}}
% {{{ [O]

\docTerm{objective-evidence}{objective evidence}{%
  Qualitative or quantitative information, records, or statements of fact
  pertaining to:
  1) The quality of an item or service;
  2) The existence and implementation of a quality system element, which is
     based on observation, measurement, or test and which can be verified.
}%

\docTerm{observation}{observation}{%
  Any statement of fact, positive or negative, made during an audit that is
  substantiated by objective evidence.
}%

\docTerm{oem-product}{oem product}{%
  Product(s) (finished devices, drug or biologics) used and/or sold by Acme
  Corporation, which are manufactured by an OEM Supplier. This includes products that
  can be used independently/standalone, or within a \gls{company} product or
  system.
}%

\docTerm{oem-supplier}{oemsupplier}{%
  Original Equipment Manufacturer - Supplier that manufactures products
  (finished devices, drugs, biologics) used and/or sold by \gls{company}, in
  which the supplier holds legal title, as well as design, manufacturing and
  regulatory responsibility.
}%

\docTerm{organizational-entity}{organizational entity}{%
  A business structure established to identify and delineate a set of resources
  and activities for an operational or strategic purpose.  Organizational
  entities typically have defined relationships to other organizational
  entities.  One organizational entity may be included in another, or interact
  in a particular way with another to perform and manage business.  Acme
  Corporation organizational entities include Corporate, Group, Business Unit and Site
  entities.
}%

\docTerm{overlabeling}{overlabeling}{%
  Original labeling is kept, and new labels, including those with translated
  versions of the product's information, are placed either partially or
  completely over the original label(s), precluding legibility.
}%

% -------------------------------------------------------------------------- }}}
% {{{ [P]

\docTerm{precaution}{precaution}{%
  A statement of a hazard alert that warns the user of a potentially hazardous
  situation which, if not avoided, may result in minor or moderate injury to the
  user or patient or damage to the equipment or other property.
}%

\docTerm{performance-goals}{performance goals}{%
  A set target established to judge whether a quality metric or objective has
  been achieved.
}%

\docTerm{performance-spec}{performance specification}{%
  The performance characteristics of a device as listed in device labeling or in
  finished product release specifications
}%

\docTerm{permanent}{permanent}{%
  Irreversible impairment or damage to a body structure or function, excluding
  trivial impairment or damage.
}%

\docTerm{policy}{policy}{%
  Document, such as the Quality Manual, that defines high-level requirements for
  the components of the Quality Management System (QMS).  These usually describe
  high-level government, industry or business requirements and the overall
  direction of the business.  Policies satisfy quality objectives.
}%

\docTerm{post-productionphase}{post-production phase}{%
  Time period in a product life cycle that begins with the manufacturing of the
  first human-use product, whether for clinical use or commercial release, and
  continues until the product is no longer in use in the market or by persons
  who may come into contact with the product.
}%

\docTerm{pre-clearance}{pre-clearance}{%
  Clearing imported Goods in advance of arrival by filing the appropriate
  documents with U.S. Customs. Ocean shipments may be pre-cleared five (5) days
  prior to arrival. Air shipments may be pre-cleared once wheels are up.
}%

\docTerm{preventive-action}{preventive action}{%
  Action taken to eliminate the causes of a potential nonconformity or other
  undesirable situation in order to prevent occurrence.
}%

\docTerm{primary-document}{primary document}{%
  Term that refers collectively to Policy, Top Level Procedure, Standard
  Operating Procedure and Work Instruction Document Categories.
}%

\docTerm{process}{process}{%
  A set of activities that transform input into expected output for the
  customers of the process.  Processes of the QMS are key components of the
  business value stream and provide the basis for the organization, management
  and use of the QMS\@.  The QMS Architecture identifies high-level Processes and
  interactions to establish the Process Structure.  A Process encompasses
  multiple QMS Documents.  QMS Processes are decomposed into Sub-processes and
  Sub-processes into Activities when warranted.
}%

\docTerm{process-control}{process control}{%
  Control of critical process variables such that they remain consistent within
  the natural process capability levels or through the use of corrective action
  permitted by process specification and defined control mechanisms that assure
  that the process will remain in a state of control.
}%

\docTerm{process-parameter}{process parameter}{%
  Any factor that defines a system and determines (or limits) its performance.
}%

\docTerm{process-structure}{process structure}{%
  A component of QMS Architecture that provides the foundation for Content,
  Governance and use.
}%

\docTerm{process-validation}{process validation}{%
  820.3(z)(1) means establishing by objective evidence that a process
  consistently produces a result or product meeting its predetermined
  specifications.
}%

\docTerm{product}{product}{%
  820.3(r) means component, manufacturing materials, in-process devices,
  finished devices, and returned devices.
  (Company) Any components, manufacturing materials, in-process devices,
  finished devices and returned devices.  An article subject to the jurisdiction
  of the Food and Drug Administration, including any food, drug and device
  intended for human or animal use, any cosmetic and biologic intended for human
  use, and any item subject to a quarantine regulation of the \Gls{fda}.
}%

\docTerm{product-change}{product change}{%
  Any modification within the scope of the Quality System to design, structure,
  intended use of a product, process or system.  A change includes creation,
  relocation or retirement of a product, equipment, process, or system.
}%

\docTerm{product-characterization}{product characterization}{%
  An exploratory phase of process development, conducted to identify sources of
  variation and to establish an operating range for process parameters as it
  relates to process outputs.
}%

\docTerm{product-family}{product family}{%
  A group of Products that share raw materials, manufacturing processes, and
  common inspection / acceptance criteria.
}%

\docTerm{product-hold}{product hold}{%
  The activity that prevents known and potential nonconforming product (finished
  devices, physical product [media] and non-physical product [media]) within
  direct and/or indirect \gls{company} control from forward movement.
}%

\docTerm{product-impact-assessment}{product impact assessment}{%
  Structured documented investigation of a product-related issue in order to
  assess potential patient safety risk and compliance risk and determine
  recommended associated actions in the field.   This activity is typically
  associated with a CAPA.
}%

\docTerm{product-in-field}{product in field}{%
  Product that is now or was once beyond a \gls{company}-controlled
  distribution center.
}%

\docTerm{product-lifecycle-process}{product lifecycle process}{%
  QMS Processes that encompass activities performed to create, deliver and
  support products across the product lifecycle and result in products supplied
  to customers.
}%

\docTerm{production-identifier}{production identifier}{%
  820.3(cc)(2) a conditional, variable portion of a UID that identifies one or
  more of the following when included on the label of the device:
  1) The lot or batch within which a device was manufactured;
  2) The serial number of a specific device;
  3) The expiration date of a specific device;
  4) The date a specific device was manufactured.
}%

\docTerm{probability}{probability}{%
  The likelihood of occurrence of harm over the life of the system taking into
  consideration the defined safety response.
}%

\docTerm{promotional-claim}{promotional claim}{%
  Any statement made or implied regarding the performance, safety, efficacy or
  quality of products - Includes comparative statements to competitive products.
}%

\docTerm{promotional-labeling}{promotional labeling}{%
  Information provided to describe marketed devices, including indications for
  use, techniques for use, benefits and advantages.  Any claims made as part of
  promotional materials will be scientifically supported and have all required
  regulatory approval.
}%

\docTerm{promotional-material}{promotional material}{%
  Any material used in support of the commercialization of a medicinal product
  or medical device.
}%

\docTerm{protocol}{protocol}{%
  1) A document that describes a study, including objective(s), design,
     methodology, statistical considerations, organization.  The protocol
     usually also gives the background and rationale for the study, but these
     could be provided in other protocol referenced documents.
  2) A prospective testing strategy plan that when executed is intended to
     produce documented evidence that the plan has been completed as define.
}%

\docTerm{purchase-order}{purchase order}{%
  A document considered legally and contractually binding, between buyer and
  seller, to purchase goods or services.
}%

% -------------------------------------------------------------------------- }}}
% {{{ [Q]

\docTerm{qms-document}{QMS document}{%
  Also referred to as Quality System Document. A term that refers collectively
  to the distinct Document Categories that make up the QMS\@. QMS documents can
  be paper or electronic, and are categorized as policies, standard operating
  procedures, work instructions or supporting material.
}%

\docTerm{qualification}{qualification}{%
  The planning, performance, and recording of tests to determine if a component
  of a manufacturing process possesses the attributes required to obtain
  a specified quality of a product. Qualification deals with components or
  elements of a process, while validation deals with an entire manufacturing
  process for a product.
}%

\docTerm{quality}{quality}{%
  820.3(s) means the total of features and characteristics that bear on the
  ability of a device to satisfy fitness-for-use, including safety and
  performance.
}%

\docTerm{quality-agreement}{quality agreement}{%
  A document that defines the requirements and responsibilities for quality-
  related activities between two entities.
}%

\docTerm{quality-audit}{quality audit}{%
  820.3(t) means a systematic, independent examination of a manufacturer’s
  quality system that is performed at defined intervals and at sufficient
  frequency to determine whether both quality system activities and the results
  of such activities comply with quality system procedures, that thee procedures
  are implemented effectively, and that these procedures are suitable to achieve
  quality system objectives.
}%

\docTerm{quality-manual}{quality manual}{%
  A QMS Document (Policy) that defines the organizational entity establishes
  to define QMS scope and structure, and outlines management oversight
  responsibilities.
}%

\docTerm{quality-manual-supplement}{quality manual supplement}{%
  A QMS Document (Policy) that defines site organizational entities, establishes
  site-specific QMS scope by defining process applicability and any associated
  regulatory exclusions or additions, and designates individuals to management
  oversight roles.
}%

\docTerm{quality-metrics}{quality metrics}{%
  An indicator used to measure performance of the organization to a specific
  quality objective.
}%

\docTerm{quality-objective}{quality objective}{%
  A higher level or systemic target the organization has established with the
  intent of accomplishment by aligning resources and actions.
}%

\docTerm{quality-policy}{quality policy}{%
  820.3(u) means the overall intentions and direction of an organizations with
  respect to quality, as established by management with executive
  responsibility.
}%

\docTerm{quality-record}{quality record}{%
  Paper or electronic evidence of planned and/or executed action. Quality
  Records are used to demonstrate conformance to applicable regulation(s) and
  other specified requirements and the operation of systems and processes.
  Quality Records (a special type of document) require control to maintain
  integrity and ensure accuracy for the required retention period of the Quality
  Record.
}%

\docTerm{quality-record-integrity}{quality record integrity}{%
  The reliability and correctness of the record; a condition that exists as long
  as accidental or intentional destruction, alteration or loss of data in the
  record does not occur.
}%

\docTerm{quality-requirements}{quality requirements}{%
  Requirements established by \gls{company} to assure Regulatory requirements
  are fully satisfied and expectations of regulators are met.
}%

\docTerm{quality-risk}{quality risk (supplier)}{%
  The actual or potential quality or reliability impact that a supplier poses to
  the user/patient.
}%

\docTerm{quality-system}{quality system}{%
  820.3(v) means the organizational structure, responsibilities, procedures,
  processes and resources for implementing quality management.
}%

% -------------------------------------------------------------------------- }}}
% {{{ [R]

\docTerm{rake}{Rake}{%
  a \gls{ruby} build system.
}%

\docTerm{readability}{readability}{%
  The ease of understanding or comprehension achieved by the style of writing.
}%

\docTerm{regression-analysis}{regression analysis}{%
  (IEEE) a software V\&V task to determine the extent of V\&V analysis and
  testing that must be prepared when changes are made to any previously examined
  software products.
}%

\docTerm{revision-control}{revision control}{%
  Management of changes to software items. Changes are usually identified by
  a numeric or alphanumeric identifier. Each revision is associated with
  a timestamp and the person making the change Revisions can be compared,
  restored, and with some types of files, merged.
}%

\docTerm{raw-materials}{raw materials}{%
  Any chemical, component, semi-finished product, subassembly, or anything that
  is not a finished product or device, but is used in its manufacture, or
  becomes part of the product itself.
}%

\docTerm{re-audit}{re-audit}{%
  An audit that occurs in response to previously identified non-conformances for
  the purpose of verifying effectiveness of audit response. The Re-audit usually
  occurs in advance of the interval or frequency of the next regularly scheduled
  audit.
}%

\docTerm{record}{record}{%
  A QMS Document Category that provides evidence of the performance and results
  of QMS Processes.  Records demonstrate conformance to requirements implemented
  in QMS Documents.
}%

\docTerm{refurbishment}{refurbishment}{%
  Activities performed on a device received from one customer prior to delivery
  to a different customer.  Such activities may include, but are not limited to
  sanitization, packaging, and correction of device conditions that assure the
  device meets specified servicing requirements, is suitable for the intended
  use, and complies with regulatory requirements.
}%

\docTerm{regulation}{regulation}{%
  Pertaining to statutes or laws enacted by the legislative branch of
  a government.
}%

\docTerm{regulatory-agency}{regulatory agency}{%
  Public authority or government agency responsible for exercising autonomous
  authority over some area of human activity in a regulatory or supervisory
  capacity.
}%

\docTerm{regulatory-authority}{regulatory authority}{%
  see Regulatory Agency.
}%

\docTerm{regulatory-opinion}{regulatory opinion}{%
  A written and approved document that provides justification, analysis, and
  risk assessment and regulatory recommendations concerning manufacturing and
  marketing products in accordance with regulatory requirements. Regulatory
  Opinions provide justification for deferment of regulatory submission for low
  regulatory risk or significance.
}%

\docTerm{regulatory-requirements}{regulatory requirements}{%
  Requirements defined by governments and agencies worldwide that \gls{company}%
  must conform with or has elected to conform with.  Regulatory Requirements
  include Regulations and Standards.
}%

\docTerm{regulatory-submission}{regulatory submission}{%
  An information package that has been assembled to be transmitted to the
  regulatory body. This preparation usually involves a step where an authorized
  agent signs the data package to attest to its accuracy and completeness.
}%

\docTerm{related-document}{related document}{%
  Term that refers collectively to Form, Record and Supporting Materials
  Document Categories.
}%

\docTerm{reliability}{readability}{%
  (IEEE) The ability of a system or component to perform its required
  functions under stated conditions for a specified period of time. See:
  software reliability.
}%

\docTerm{remanufacturer}{remanufacturer}{%
  Any person/business who processes, conditions, renovates, repackages, restores
  or does any other act to a finished device that significantly changes the
  finished device's performance or safety specifications or intended use.
}%

\docTerm{remanufacturing}{remanufacturing}{%
  Activities which process, condition, renovate, repackage, restore or do any
  other act to a finished device that significantly changes the finished
  device's performance or safety specifications or intended use.
}%

\docTerm{remedialaction}{remedial action}{%
  Any action, other than routine maintenance or servicing of a device, necessary
  to prevent recurrence of a reportable event.
}%

\docTerm{remediationplan}{remediation plan}{%
  A plan describing the action required to eliminate the cause(s) of an existing
  or potential non-conformance/exception or other undesirable situation to
  prevent recurrence or occurrence.
}%

\docTerm{removal}{removal}{%
  Physical removal of a device from its point of use to some other location for
  repair, modification, adjustment, re-labeling, destruction or inspection.
}%

\docTerm{report}{report}{%
  A display of data / information used for creating awareness and / or decision
  making.
}%

\docTerm{reportable-event}{reportable event}{%
  An event that must be reported to regulatory agencies because it reasonably
  suggests that a marketed device may have caused or contributed to a death or
  serious injury or has malfunctioned and that the device or similar device
  marketed by the manufacturer would be likely to cause or contribute to a death
  or serious injury if the malfunction were to recur.
}%

\docTerm{reprocessing}{reprocessing}{%
  An activity performed before a device is distributed, which may include
  repackaging, retesting, re-sterilization or re-inspecting WIP\@.  The activity
  is documented in the DHR.
}%

\docTerm{requirement}{requirement}{%
  Need or expectation that is stated, generally implied, or obligatory.
}%

\docTerm{requirements-matrix}{requirements matrix}{%
  A QMS Document that maps external requirements to QMS Processes and
  Sub-processes.  External requirements include Regulations, Standards and
  Corporate Policies.
}%

\docTerm{residual-risk}{residual risk}{%
  Risk remaining after protective measures have been taken.
}%

\docTerm{rework}{rework}{%
  820.3(x) means action taken on a nonconforming product so that it will;
  fulfill the specified DMR requirements before it is released for
  distribution.
}%

\docTerm{risk}{risk}{%
  Combination of the probability of occurrence of harm and the severity of that
  harm.
}%

\docTerm{risk-analysis}{risk analysis}{%
  The systematic use of available information to identify hazards and to
  estimate the risk of the identified hazards.
}%

\docTerm{risk-benefit-summary}{risk benefit summary}{%
  Rationale for not mitigating a risk due to its feasibility and benefit to
  society of continued use outweighing the risk.
}%

\docTerm{risk-control}{risk control}{%
  1) Process in which decisions are made and measures implemented by which risks
     are reduced to, or maintained within, specified levels.
  2) Specific requirements that are implemented to redact the severity of the
     harm or the probability of its occurrence, or both.
}%

\docTerm{risk-estimation}{risk estimation}{%
  Process used to assign values to the probability of occurrence of harm and the
  severity of that harm.
}%

\docTerm{risk-evaluation}{risk evaluation}{%
  Comparing the estimated risk against given risk criteria to determine the
  acceptability of the risk.
}%

\docTerm{risk-management}{risk management}{%
  Systematic application of management policies, procedures and practices to the
  tasks of analyzing, evaluating and controlling (mitigating) risk.  Risk
  Management is an Enabling Process.
}%

\docTerm{risk-management-file}{risk management file}{%
  Set of records and other documents (not necessarily contiguous) that is
  produced by risk management (process).
}%

\docTerm{robustness}{robustness}{%
  The degree to which a software system or component can function correctly
  in the presence of invalid inputs or stressful environmental conditions.
  See: software reliability.
}%

\docTerm{root-cause}{root cause}{%
  A fundamental reason for an issue that could result in a nonconformity.
  Elimination of a root cause prevents the issue from occurring.
}%

\docTerm{root-cause-analysis}{root cause analysis}{%
  Documented analysis necessary to determine the original or true cause of
  a system, product or process non-conformance; this effort extends beyond the
  effects of a problem to discover its most fundamental cause.
}%

\docTerm{routine-servicing}{routine servicing}{%
  Any regularly scheduled maintenance of a device, including the replacement of
  parts at the end of their normal life expectancy, e.g., calibration,
  replacement of batteries, and responses to normal wear and tear.
}%

\docTerm{rspec}{rspec}{%
  A \gls{rubygem} that provides behavior driven test development.
}%

\docTerm{ruby}{ruby}{%
  (WikipediA) An interpreted, high-level, general purpose programming language.
  Ruby is dynamically typed and garbage-collected.  It support multiple
  programming paradigms, including procedural, object-oriented, and functional
  programming.
}%

\docTerm{rubygem}{RubyGem}{%
  (Wikipedia) A package manager for the \gls{ruby} programming language that
  provides a standard format for distributing \gls{ruby} programs and libraries
  (in a self-contained format called a `gem'), a tool designed to easily manage
  the installation of gems, and a server for distributing them.
}%

% -------------------------------------------------------------------------- }}}
% {{{ [S]

\docTerm{safety}{safety}{%
  Freedom from unacceptable risk.
}%

\docTerm{segregate}{segregate}{%
  To separate or isolate by physical or electronic means to control product with
  appropriate identification of status.
}%

\docTerm{serious-injury}{serious injury}{%
  An injury or illness that is life-threatening, even if temporary; results in
  permanent impairment of a body function or permanent damage to body structure;
  or necessitates medical or surgical intervention to preclude permanent
  impairment of a body function or permanent damage to a body structure.
}%

\docTerm{severity}{severity}{%
  A measure of the possible consequences of a hazard.
}%

\docTerm{service}{service}{%
  An activity performed for, or on behalf of, \gls{company} that supports the
  production or distribution function in the organization.
}%

\docTerm{service-family}{service family}{%
  An inclusive term meaning any group of Services that share service realization
  processes and acceptance requirements.
}%

\docTerm{signature}{signature}{%
  One's name as written by oneself used to identify the person signing
  a document. An electronic signature is the electronic equivalent to
  a hand-written signature.
}%

\docTerm{site}{site}{%
  A \gls{company} organizational entity characterized by location, consisting
  of one or more buildings identified to focus on processes related to specific
  technologies/products or geographies.  Sites are typically associated with
  unique \Gls{fda} registrations and/or notified body certifications.
}%

\docTerm{software-control}{software control}{%
  Software Control Tags (SCT) are used to indicate a risk control is referenced
  in the Mode of Control (MOC) table and Software Requirements Specification
  (SRS).
}%

\docTerm{software-inspection}{software inspection}{%
  Inspection in software engineering, refers to peer review of any work product
  by trained individuals who look for defects using a well defined process.
}%

\docTerm{software-reliability}{software reliability}{%
  (IEEE)
  1) the probability that software will not cause the failure of a
  system for a specified time under specified conditions. The probability is
  a function of the inputs to and use of the system in the software. The
  inputs to the system determine whether existing faults, if any, are
  encountered.
  2) The ability of a program to perform its required functions
  accurately and reproducibly under stated conditions for a specified period of time.
}%

\docTerm{software-risk-management}{software risk management}{%
  Systematic application of policies, procedures, and practices to the task of
  analyzing, evaluating, and controlling risk associated with software
  systems.
}%

\docTerm{software-system}{software system}{%
  Integrated collection of software items organized to accomplish a specific
  function or a set of functions. (A software system can be decomposed into one
  or more software items, and each software item is composed of one or more
  software units or decomposable software items. The definition and granularity
  of the software items and software units is left to the manufacturer.) [IEC
  62304].
}%

\docTerm{software-validation}{software validation}{%
  Software Validation confirms through objective evidence that software
  specifications conform to user needs and intended uses, and that the
  particular requirements implemented through software can be consistently
  fulfilled.
}%

\docTerm{software-verification}{software verification}{%
  1) Confirmation by examination and provision of objective evidence that
     specified requirements have been fulfilled through software-level testing;
  2) The process of determining whether or not the products of a given phase of
     the software development cycle fulfill the requirements established during
     the previous phase; and
  3) The demonstration of consistency, completeness, and correctness of the
     software at a given stage, achieved through tools such as code inspection,
     analysis, unit testing, and integration testing.
}%

\docTerm{sourcing}{sourcing}{%
  The process of controlling the acquisition of materials or services to be used
  in a product or a Product Lifecycle process.  Sourcing is a Product Lifecycle
  Process.
}%

\docTerm{specification}{specification}{%
  820.3(y) means any requirement with which a product, process, service or other
  activity must conform.
}%

\docTerm{standard}{standard}{%
  Agreed upon set of rules that defines how people develop and manage products.
}%

\docTerm{static-analysis}{static analysis}{%
  1) (NSB) Analysis of a program that is performed without executing the
     program.
  2) (IEEE) The process of evaluating a system or component based on its form,
     structure, content, documentation.
}%

\docTerm{sub-process}{sub-process}{%
  A decomposition of a QMS Process due to scale or complexity, or to achieve
  usability goals.  Sub-processes are decomposed into Activities when warranted.
}%

\docTerm{sub-tier-supplier}{sub-tier supplier}{%
  An organization or business entity that provides Products or Services to
  a Supplier.
}%

\docTerm{supplier}{supplier}{%
  An organization or business entity that a \gls{company} facility does
  business with for the provision of Products or Services. This business entity
  could be another \gls{company} facility.
}%

\docTerm{supply-risk}{supply risk}{%
  The actual or potential business impact that a supplier poses to Acme
  Corporation.
}%

\docTerm{supporting-material}{supporting material}{%
  Document that is used as a tool to aid in the execution of business processes
  (e.g. Training Matrices, User Guides, etc.).
}%

\docTerm{system-enhancement}{system enhancement}{%
  Changes to a system to add new requirements, revise requirements and/ or
  delete requirements for an existing product system.
}%

\docTerm{system-maintenance}{system maintenance}{%
  Changes to a system designed to sustain its intended use.  System maintenance
  involves no new, revised or deleted requirements.
}%

\docTerm{system-retirement}{system retirement}{%
  Scheduled removal of a system or system component from production use.
}%

\docTerm{system-validation}{system validation}{%
  1) The process of evaluating software at the end of the software development
     process, through integrated systems-level testing, to determine the
     correctness of the software with respect to the user needs and
     requirements.  May be accomplished through processes such as system-level
     software end-phase development testing, Alpha testing, or Beta testing,
     either alone or as part of a total Systems Validation effort.
  2) A set of detailed procedures and activities that demonstrate and document
     a system (hardware and software) functions properly in a given operating
     environment and that the system contains features prescribed in its
     requirements and system specifications.  Validation also demonstrates the
     system does not produce unexpected results during normal use or high load
     situations.
}%

% -------------------------------------------------------------------------- }}}
% {{{ [T]

\docTerm{technical-design-inputs}{technical design inputs}{%
  see System Requirements Specification.
}%

\docTerm{technical-file}{technical file}{%
  A summary of the technical documentation for a device or device product family
  that is issued and maintained by RA Europe.
}%

\docTerm{testing}{testing}{%
  (IEEE)
  1) The process of operating a system or component under specified
  conditions, observing or recording the results, and making an evaluation of
  some aspect of the system or component.
  2) The process of analyzing a software item to detect the differences between
  existing and required conditions, i.e. bugs, and to evaluate the features of
  the software item.
}%

\docTerm{testing-acceptance}{acceptance testing}{%
  (IEEE) Testing conducted to determine whether or not a system satisfies its
  acceptance criteria and to enable the customer to determine whether or not
  to accept the system.
}%

\docTerm{testing-functional}{functional testing}{%
  (IEEE)
  1) Testing that ignores the internal mechanism or structure of a
  system or component and focuses on the outputs generated in response to
  selected inputs and execution conditions.
  2) Testing conducted to evaluate the compliance of a system or component
  with specified functional requirements and corresponding predicted results.
  Syn: black-box testing, input/output driven testing. Contrast with testing,
  structural.
}%

\docTerm{testing-integration}{integration testing}{%
  (IEEE) An orderly progression of testing in which software elements,
  hardware elements, or both are combined and tested, to evaluate their
  interactions, until the entire system has been integrated.
}%

\docTerm{testing-interface}{interface testing}{%
  (IEEE) Testing conducted to evaluate whether systems or components pass
  data and control correctly to one another. Contrast with testing, unit;
  testing, system.
}%

\docTerm{testing-performance}{performance testing}{%
  (IEEE) Functional testing conducted to evaluate the compliance of a system
  or component with specified performance requirements.
}%

\docTerm{testing-system}{system testing}{%
  (IEEE) The process of testing an integrated hardware and software system to
  verify that the system meets its specified requirements. Such testing may
  be conducted in both the development environment and the target
  environment.
}%

\docTerm{testing-unit}{unit testing}{%
  1) (NIST) Testing of a module for typographic, syntactic, and logical
  errors, for correct implementation of its design, and for satisfaction of
  its requirements.
  2) (IEEE) Testing conducted to verify the implementation
  of the design for one software element; e.g., a unit or module; or a
  collection of software elements.
}%

\docTerm{testing-usability}{usability testing}{%
  Tests designed to evaluate the machine/user interface. Are the
  communication device(s) designed in a manner such that the information is
  displayed in a understandable fashion enabling the operator to correctly
  interact with the system?
}%

\docTerm{testing-regression}{regression testing}{%
  1) (NIST) Rerunning test cases which a program has previously executed
     correctly in order to detect errors spawned by changes or corrections made
     during software development and maintenance.
  2) The running of test cases in order to detect potential impact errors
     created by changes or corrections made to code during the software
     development cycle.
}%

\docTerm{testing-stress}{stress testing}{%
  (IEEE) Testing conducted to evaluate a system or component at or beyond the
  limits of its specified requirements.
}%

\docTerm{third-party-service-provider}{third party service provider}{%
  Functions, sites, or individuals that are not affiliated with or directly
  managed by \gls{company}.
}%

\docTerm{tlc-article}{tlc-article}{%
  A \LaTeX\ document class that orchestrates a logical arrangement for document
  header, footer, contents, and margins.
}%

\docTerm{trend-analysis}{trend analysis}{%
  A method for comparing past and present data to monitor changes in a process;
  trend analysis can provide warning of unfavorable conditions, which should be
  corrected or favorable conditions, which should be studied for possible
  permanent improvement of the process.
}%

\docTerm{traceability}{traceability}{%
  (IEEE)
  1) The degree to which a relationship can be established between two or
  more products of the development process, especially products having a
  predecessor-successor or master-subordinate relationship to one another;
  e.g., the degree to which the requirements and design of a given software
  component match.
  2) The degree to which each element in a software development product
  establishes its reason for existing; e.g., the degree to which each element
  in a bubble chart references the requirement that it satisfies.
}%

% -------------------------------------------------------------------------- }}}
% {{{ [U]

\docTerm{use-error}{use error}{%
  Pattern of use failure that indicates a failure mode that is likely to occur
  with use and thus has a reasonable possibility of predictability of
  occurrence.
}%

\docTerm{user-error} {user error}{%
  Unusual pattern of failure that indicates a failure mode resulting from
  fundamental errors by humans and that has no reasonable possibility of being
  predicted.
}%

% -------------------------------------------------------------------------- }}}
% {{{ [V]

\docTerm{validation}{validation}{%
  820.3(z) means confirmation by examination and provision of objective evidence
  that the particular requirements for a specific intended use can be
  consistently fulfilled.
  (company) Validation confirms through examination and provision of objective
  evidence that the particular requirements for a specific intended use can be
  consistently fulfilled.  Design Validation provides objective evidence that
  the device specifications conform to user needs and intended use(s).  Process
  Validation means establishing by objective evidence that a process
  consistently produces a result or product meeting its predetermined
  specifications.  Software Validation confirms through objective evidence that
  software specifications conform to user needs and intended uses, and that the
  particular requirements implemented through software can be consistently
  fulfilled.
}%

\docTerm{verification}{verification}{%
  820.3(aa) means confirmation by examination and provision of objective
  evidence the specified requirements have been fulfilled.
}%

% -------------------------------------------------------------------------- }}}
% {{{ [W]

\docTerm{warning}{warning}{%
  An alert to the user about a situation which, if not avoided, could result in
  death or serious injury.  Note: The distinction between warnings and
  precautions is a matter of degree of likelihood and seriousness of the
  hazard.
}%

\docTerm{watir}{watir}{%
  A \gls{ruby} Gem that is used to interact with a browser.
}%

\docTerm{where-appropriate}{where appropriate}{%
  Deemed to be required unless appropriate rationale is documented.
}%

% ------------------------------------------------------------------------- }}
