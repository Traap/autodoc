\subsection{Test Plan Instructions}
This Test Plan describes Test Suits, Test Cases, and Test Steps that demonstrate
how the Configuration Item satisfies the IUR.  Each Test Plan describes any
setup criteria needed to conduct the Test Steps. Each Test Plan contains a list
of IUR and the Test Steps that demonstrate how the Configuration Item satisfies
the IUR. Each Test Step is marked passed or failed as it is completed.  Each
Test Plan is marked passed when all Test Steps pass or failed if a single Test
Step fails.  This serves as a record of completed Test Plans and Test Steps.

Test Plans are automatically run by the computer generating a report in PDF
format.  Test Plans are code-reviewed prior to use using the same code-review
practice that is used for the medical device software.  Test Reports are stored
in the Configuration Management tool that is used by the medical device
software.  Test Steps that fail during computer execution or observations made
during document routing, result in tickets being created to address the issue.
This results in another execution of the Report Package and document routing.

When it becomes necessary to annotate a computer generated document, a Company's
documentation practices must be followed.  The documentation practices are
summarized below:

\begin{enumerate}
\item When necessary to write down test values, they are written using
an ink pen.

\item Any cross-outs of results or modifications to any test step shall be
  initialed and dated.

\item All appropriate blanks shall be filled in or annotated as ``N/A'' (not
applicable).

\item No whiteout or other text-obscuring technique is to be used.

\item Any annotation must be signed and dated.

\end{enumerate}

In most cases, a Test Plan is to be judged as "fail" if any Test Step within
that Test Plan fails. If the failed Test Step does not impact proper
evaluation of the safety and effectiveness of the device, the Test Plan in
which the Test Step failed may be judged "pass".  For such failed Test Steps a
rationale will be noted in the comments section and in the final Report
Package.  If a failed Test Plan is re-executed a record of the failed
results are included with the Test Plan when it passes and attached to the
Test Plan.  If a Test Plan is re-executed a new report is
created and routed for approval in the document management system.

\subsection{Test Plan Storage and Review}
This Test Plan is part of a Company's automated validation framework.  The
framework consists of following parts:
\begin{description}
  \item[\LaTeX] files are used to provide an Abstract, Introduction, Intended
    Use Requirements, Test Plan Overview, Test Equipment, Configuration
    Item Validation, Conclusion, and Change Summary.

  \item[Ruby] software is used to run the automated framework to collect test
    evidence.

  \item[Git] is used as the storage repository for \LaTeX \& Yaml files.

  \item[Git] (https://github.com/Traap/autodoc) is used to create a pull-request
    to review the \LaTeX \& YAML files prior to use.

  \item[Evidence] Test Plan output includes one Test Suite, Test Plan, and Test
    Step, and Test Evidence.
\end{description}
