\subsection{Test Plan Instructions}
This Test Plan describes Test Suits, Test Cases, and Test Steps that demonstrate
how the Configuration Item satisfies the IUR.  Each Test Plan describes any
setup criteria needed to conduct the test.  Each Test Plan contains a list of
IUR\textbackslash{s} and the steps that demonstrate how the Configuration Item
satisfies the IUR.  Each Test Step is marked passed or failed as it is
completed.  Each Test Plan is marked passed when all Test Steps pass or failed
if a single Test Step fails.  Failures are addressed per \sopISS. This serves as
a record of the completed
test.

Test Plans are automatically run by the computer, generating a report in PDF
format.  This Report Package is reviewed prior to execution per \sopSDLC.  The
Report Package is routed and archived in the Quality Management System.  When it
becomes necessary to annotate a computer generated document \sopGDP\ must be
followed.

\subsection{Test Plan Storage and Review}
This Test Plan is part of a Company's automated validation framework.  The
framework consists of following parts:
\begin{description}
  \item[\LaTeX] files are used to provide an Abstract, Introduction, Intended
    Use Requirements, Test Plan Overview, Test Equipment, Configuration
    Item Validation, Conclusion, and Change Summary.  \LaTeX\ files are converted
    assembled into PDF documents.  PDF documents are routed using the Company's
    document management system for approval.

  \item[Ruby] software is used to run the automated framework to collect test
    evidence.

  \item[Git] is used as the storage repository for \LaTeX\ \& YAML files, a Git
    pull-request is used to review the \LaTeX\ \& YAML files prior to use.

  \item[Evidence] Test Plan output includes one Test Suite, Test Plan, and Test
    Step, and Test Evidence.

  \item[YAML] files define the Test Plan, Test Suite, and Test Steps that are
    processed to generate test evidence.
\end{description}
