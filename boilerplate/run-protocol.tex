%%%%%%%%%%%%%%%%%%%%%%%%%%%%%%%%%%%%%%%%%%%%%%%%%%%%%%%%%%%%%%%%%%%%%%%%%%%%%%%%
%% GLOBAL VARIABLES
%%%%%%%%%%%%%%%%%%%%%%%%%%%%%%%%%%%%%%%%%%%%%%%%%%%%%%%%%%%%%%%%%%%%%%%%%%%%%%%%
\edef\protocolpath{src}                %% path to protocol src files

%%%%%%%%%%%%%%%%%%%%%%%%%%%%%%%%%%%%%%%%%%%%%%%%%%%%%%%%%%%%%%%%%%%%%%%%%%%%%%%%
%% FUNCTION : runProtocol
%% ARGUMENTS:
%%   #1 - the protocol subfolder that has the test-steps.sh script file to run
%% DESCRIPTION: Perform a shell exit to run all the steps in a given protocol.
%% The standard out and standard error are logged for each protocol step and
%% sent to the step-n.log file, where n is the protocol step number.   The
%% pass/fail status of each step is sent to the step-n.tex file.   The pass/fail
%% status of the entire protocol is sent to the recipe-results-99.tex file.
%%%%%%%%%%%%%%%%%%%%%%%%%%%%%%%%%%%%%%%%%%%%%%%%%%%%%%%%%%%%%%%%%%%%%%%%%%%%%%%%
\newcommand{\runProtocol}[1]{
    \IfFileExists{\protocolpath/#1/test-steps.sh}
       {
       %% shell exit to create the output folder
       \immediate\write18{mkdir -p test-output/protocol/#1/}
       %% Initialize the allpassed variable to true.  This variable will be
       %% used to accumulate a bash command that will write the pass/fail
       %% result of the protocol to a file.
       \global\edef\allpassed{true}
       \csvreader[no head, respect all, separator=semicolon]
           {\protocolpath/#1/test-steps.sh}
           {1=\stepCmd}
           { %% Strip the trailing # symbol from the command string.
             \StrGobbleRight{\stepCmd}{1}[\cmdstr]
             %% fully expand the \thecsvrow macro so that it can be used in
             %% an edef.
             \StrExpand{\thecsvrow}{\stepno}
             %% Create some local variables for storing path information
             %% so that it can be reused in construction of the bash
             %% command strings
             \edef\outpath{test-output/protocol/#1}
             \edef\resultfile{\outpath/step-\stepno.tex}
             %% Construct the command string, logging the stdout and error,
             %% and writing the pass/fail result to a file.
             \edef\cmdstr{\cmdstr &> \outpath/step-\stepno.log}
             \edef\cmdstr{\cmdstr && echo PASS > \resultfile }
             \edef\cmdstr{\cmdstr || echo FAIL > \resultfile }
             \edef\cmdstr{bash -c "\cmdstr"}
             \immediate\write18{\cmdstr}  %% RUN THE COMMAND
             %% accumulate the pass/fail result into the allpassed command.
             \global\edef\allpassed{ \allpassed && grep PASS \resultfile }
           }
           %% Add a failure case to the allpassed command, and set the
           %% output file.
           \global\edef\allpassed{ \allpassed || grep FAIL \resultfile }
           \global\edef\allpassed{ (\allpassed)> \outpath/recipe-results-99.tex }
           \global\edef\allpassed{ bash -c "\allpassed" }
           \immediate\write18{ \allpassed } %% EXECUTE ALLPASSED COMMAND.
       }{}
}


%%%%%%%%%%%%%%%%%%%%%%%%%%%%%%%%%%%%%%%%%%%%%%%%%%%%%%%%%%%%%%%%%%%%%%%%%%%%%%%%
%% FUNCTION: genericProtocol
%% DERSCRIPTION: Create a protocol subsection for an arbitrary protocol.
%%    This function takes 5 arguments: a basename, a protocol title, and three
%%    macros for creating the protocol overview, the test steps subsection and
%%    the test evidence subsection.   The protocol title is used in the section
%%    header and should be distinct from all other protocol titles to ensure
%%    distinct entries in the TOC.   The basename is used to locate the protocol
%%    source files.
%%
%%    This function is used by both the automatedProtocol and manualProtocol
%%    functions.
%%
%%    The function will display display an error message if the protocol
%%    description.tex file cannot be found.
%%%%%%%%%%%%%%%%%%%%%%%%%%%%%%%%%%%%%%%%%%%%%%%%%%%%%%%%%%%%%%%%%%%%%%%%%%%%%%%%
\newcommand{\genericProtocol}[5]{
    \newpage
    \subsection{ Protocol - #2 }
    #3{#1}{#2}
    \subsubsection{Test Steps}
    #4{#1}{#2}
    \newpage
    \subsubsection{Test Evidence}
    #5{#1}
}

%%%%%%%%%%%%%%%%%%%%%%%%%%%%%%%%%%%%%%%%%%%%%%%%%%%%%%%%%%%%%%%%%%%%%%%%%%%%%%%%
%% FUNCTION: protocolDescription
%% DERSCRIPTION: Create a description subsection for a protocol.   This function
%%    takes two arguments: a basename and a protocol title.   The
%%    protocol title is used in the section header and should be distinct from
%%    all other protocol titles to ensure distinct entries in the TOC.   The
%%    basename is used to locate the protocol source files.
%%
%%    This function is used by both the automatedProtocol and manualProtocol
%%    functions.
%%
%%    The function will display display an error message if the protocol
%%    description.tex file cannot be found.
%%%%%%%%%%%%%%%%%%%%%%%%%%%%%%%%%%%%%%%%%%%%%%%%%%%%%%%%%%%%%%%%%%%%%%%%%%%%%%%%
\newcommand{\protocolDescription}[2]{
    \IfFileExists{\protocolpath/#1/Overview.tex}
        {\input{\protocolpath/#1/Overview.tex}}
        {\textcolor{red}{ERROR: protocol overview file for #2 not found.
           Please create a \protocolpath/#1/protocol.tex file that describes
           intent of this protocol.}}
}

%%%%%%%%%%%%%%%%%%%%%%%%%%%%%%%%%%%%%%%%%%%%%%%%%%%%%%%%%%%%%%%%%%%%%%%%%%%%%%%%
%% FUNCTION: genericTestSteps
%% ARGUMENTS
%%   #1 The subfolder in which the test-steps.sh file is located
%%   #2 The macro used to generate the test steps table.
%% DERSCRIPTION: Create the test steps subsection for an arbitrary protocol.
%%    The function takes 3 arguments: a basename and a protocol title and a
%%    macro for creating the steps table.
%%%%%%%%%%%%%%%%%%%%%%%%%%%%%%%%%%%%%%%%%%%%%%%%%%%%%%%%%%%%%%%%%%%%%%%%%%%%%%%%
\newcommand{\genericTestSteps}[2]{
    \inputIfExists{\protocolpath/#1/test-steps.tex}
    \IfFileExists{\protocolpath/#1/test-steps.sh}{#2{#1}}{}
}

%%%%%%%%%%%%%%%%%%%%%%%%%%%%%%%%%%%%%%%%%%%%%%%%%%%%%%%%%%%%%%%%%%%%%%%%%%%%%%%
%% FUNCTION: manualStepResult
%% ARGUMENTS
%%  #1 - unused
%%  #2 - unused
%% DESCRIPTION: Map any argument values to the literal PASS.  This is used
%% to populate the pass/fail result for manual tests.
%%%%%%%%%%%%%%%%%%%%%%%%%%%%%%%%%%%%%%%%%%%%%%%%%%%%%%%%%%%%%%%%%%%%%%%%%%%%%%%
\newcommand{\manualStepResult}[2]{ PASS }
%%%%%%%%%%%%%%%%%%%%%%%%%%%%%%%%%%%%%%%%%%%%%%%%%%%%%%%%%%%%%%%%%%%%%%%%%%%%%%%
%% FUNCTION: automatedStepResult
%% ARGUMENTS
%%  #1 - the subfolder in which the protocol results are stored.
%%  #2 - the step number of the protocol step.
%% DESCRIPTION: import the step's pass/fail results from a file.   If the
%% file cannot be found, instead return the literal MISSING.
%%%%%%%%%%%%%%%%%%%%%%%%%%%%%%%%%%%%%%%%%%%%%%%%%%%%%%%%%%%%%%%%%%%%%%%%%%%%%%%
\newcommand{\automatedStepResult}[2]{
   \IfFileExists{test-output/protocol/#1/step-#2.tex}
      {\input{test-output/protocol/#1/step-#2.tex}}
      {\textcolor{gray}{MISSING}}
}

%%%%%%%%%%%%%%%%%%%%%%%%%%%%%%%%%%%%%%%%%%%%%%%%%%%%%%%%%%%%%%%%%%%%%%%%%%%%%%%
%% FUNCTION: stepsTable
%% ARGUMENTS
%%  #1 - the subfolder in which the protocol results are stored.
%%  #2 - unused
%% DESCRIPTION: loop through the rows in a test-steps.sh script file and create
%% a table from the metadata.   If the file cannot be found, then display a
%% placeholder indicating that the data is missing.
%%%%%%%%%%%%%%%%%%%%%%%%%%%%%%%%%%%%%%%%%%%%%%%%%%%%%%%%%%%%%%%%%%%%%%%%%%%%%%%
\newcommand{\stepsTable}[2]{
    \IfFileExists{\protocolpath/#1/test-steps.sh}{
    \csvreader[no head, separator=semicolon, respect all,
        longtable=|c|L{1.2cm}|L{10cm}|C{2cm}|,
        table head= \hline Step & IUR & Confirm / Expectation & Result \endhead \hline ,
        late after line=\\\hline]
        { \protocolpath/#1/test-steps.sh}
        { 2=\stepIUR, 3=\stepDescription, 4=\stepExpectedResult }
        { \thecsvrow & \stepIUR & \stepDescription \newline \stepExpectedResult & #2{#1}{\thecsvrow} }
    }{
    \textcolor{red}{ERROR: \protocolpath/#1/test-steps.sh could not be found.
    Please a bash script file containing the bash commands to execute.

    each row of the test-steps.sh file should be of the form:

    \framebox{ cmd \#; IUR; purpose ; expectation ; }

    where cmd is the bash command to execute, and IUR, purpose, and expectation
    are metadata used for populating the summary table. }
    }
}

%%%%%%%%%%%%%%%%%%%%%%%%%%%%%%%%%%%%%%%%%%%%%%%%%%%%%%%%%%%%%%%%%%%%%%%%%%%%%%%
%% The following commands provide the utility to create tables that summarize
%% the information in a test-steps.sh script.   The noStepsTable is semantically
%% equivalent to a NOOP, while manualStepsTable and automatedSteps table
%% are partially evaluated forms of the stepsTable with the appropriate
%% functional parameter for displaying the pass/fail result of a test step.
%% ARGUMENTS
%%  #1 - the subfolder in which the input and output files for the protocol are
%%       stored.
%%%%%%%%%%%%%%%%%%%%%%%%%%%%%%%%%%%%%%%%%%%%%%%%%%%%%%%%%%%%%%%%%%%%%%%%%%%%%%%
\newcommand{\noStepsTable}[1]{}
\newcommand{\manualStepsTable}[1]{\stepsTable{#1}{\manualStepResult}}
\newcommand{\automatedStepsTable}[1]{\stepsTable{#1}{\automatedStepResult}}


%%%%%%%%%%%%%%%%%%%%%%%%%%%%%%%%%%%%%%%%%%%%%%%%%%%%%%%%%%%%%%%%%%%%%%%%%%%%%%%
%% FUNCTION: manualTestResult
%% ARGUMENTS
%%  #1 - The subfolder in which the output files for the protocol are stored.
%% DESCRIPTION: Import the Additional-results.tex file containing the test
%% evidence for a given protocol step.   If the file cannot be found, then
%% provide a useful error message.
%%%%%%%%%%%%%%%%%%%%%%%%%%%%%%%%%%%%%%%%%%%%%%%%%%%%%%%%%%%%%%%%%%%%%%%%%%%%%%%
\newcommand{\manualTestEvidence}[1]{
   \IfFileExists{\protocolpath/#1/Additional-Results.tex}
      {\input{\protocolpath/#1/Additional-Results.tex}}
      {\textcolor{red}{ERROR: Could not find \protocolpath/#1/Additional-Results.tex.
      For manual tests you must include an Additional-Results file that
      provides the test evidence for the protocol.}
      }
}

%%%%%%%%%%%%%%%%%%%%%%%%%%%%%%%%%%%%%%%%%%%%%%%%%%%%%%%%%%%%%%%%%%%%%%%%%%%%%%%
%% FUNCTION: automatedTestResult
%% ARGUMENTS
%%  #1 - The subfolder in which the output files for the protocol are stored.
%% DESCRIPTION: Traverse the test-steps.sh file for a given protocol and import
%% the test evidence for each step in the protocol.  If the test-steps.sh or
%% evidence files cannot be found then display a helpful error message.
%%%%%%%%%%%%%%%%%%%%%%%%%%%%%%%%%%%%%%%%%%%%%%%%%%%%%%%%%%%%%%%%%%%%%%%%%%%%%%%
\newcommand{\automatedTestEvidence}[1]{
    \IfFileExists{\protocolpath/#1/test-steps.sh}{
    %% Traverse the test-steps.sh file, extracting the metadata for each
    %% test step, and create a summary table and results detail for each step.
        \csvreader[no head, separator=semicolon, respect all]
            { \protocolpath/#1/test-steps.sh}
            { 1=\stepCmd,2=\stepIUR,3=\stepDescription,4=\stepExpectedResult }
            { %% Summarize the test step's metadata
              \begin{tabular}{ll}
                Test Step: & \thecsvrow \\
                Requirement: & \stepIUR \\
                Confirm: & \stepDescription \\
                Expectation: & \stepExpectedResult \\
                Command: & \StrGobbleRight{\stepCmd}{1} \\
                Test Result: &
                    \IfFileExists{test-output/protocol/#1/step-\thecsvrow.tex}
                        { \input{test-output/protocol/#1/step-\thecsvrow.tex}}
                        { \textcolor{gray}{MISSING}}
              \end{tabular} \\
              %% import the test result details.
              \IfFileExists{test-output/protocol/#1/step-\thecsvrow.log}
                { Evidence: Starts on next line.
                    \lstset{language=[],extendedchars=true}
                    \lstinputlisting{test-output/protocol/#1/step-\thecsvrow.log}
              %% if the results could not be found then display a error message
                }{ \placeholder\\ }
            }
    }{  %% if test-steps.sh could not be found then display an error message.
        \textcolor{red}{
            Protocol could not be executed because no test-steps.sh
            script was provided. \\
        }
    }
}


%%%%%%%%%%%%%%%%%%%%%%%%%%%%%%%%%%%%%%%%%%%%%%%%%%%%%%%%%%%%%%%%%%%%%%%%%%%%%%%%
%% FUNCTION: manualTestSteps
%% ARGUMENTS:
%%   #1 the subfolder in which the  test-steps.sh file is located
%%   #2 the name of the protocol
%% DERSCRIPTION: Create the test steps subsection for a manual protocol.
%%    The function will display display an error message if neither a
%%    protocols/basename/test-steps.sh nor protocols/basename/test-steps.tex
%%    exist.
%%%%%%%%%%%%%%%%%%%%%%%%%%%%%%%%%%%%%%%%%%%%%%%%%%%%%%%%%%%%%%%%%%%%%%%%%%%%%%%%
\newcommand{\manualTestSteps}[2]{
    \IfFileExists{\protocolpath/#1/test-steps.sh}
        {\genericTestSteps{#1}{\manualStepsTable}}
        {\IfFileExists{\protocolpath/#1/test-steps.tex}
            {\genericTestSteps{#1}{\noStepsTable}}
            {\textcolor{red}{ERROR: protocol test-steps file for #2 not found.
               Please create a \protocolpath/#1/test-steps.sh or \protocolpath/#1/test-steps.tex
               file that describes the test steps for this protocol.}
            }
        }
}

%%%%%%%%%%%%%%%%%%%%%%%%%%%%%%%%%%%%%%%%%%%%%%%%%%%%%%%%%%%%%%%%%%%%%%%%%%%%%%%%
%% FUNCTION: automatedTestSteps
%%   #1 the subfolder in which the  test-steps.sh file is located
%%   #2 the name of the protocol
%% DERSCRIPTION: Create the test steps subsection for a automated protocol.
%%    The function will display display an error message if neither a
%%    protocols/basename/test-steps.sh nor protocols/basename/test-steps.tex
%%    exist.
%%%%%%%%%%%%%%%%%%%%%%%%%%%%%%%%%%%%%%%%%%%%%%%%%%%%%%%%%%%%%%%%%%%%%%%%%%%%%%%%
\newcommand{\automatedTestSteps}[2]{
    \IfFileExists{\protocolpath/#1/test-steps.sh}
        {\genericTestSteps{#1}{\automatedStepsTable}}
        {\textcolor{red}{ERROR: protocol test-steps file for #2 not found.
               Please create a \protocolpath/#1/test-steps.sh
               file that describes the test steps for this protocol.}
        }
}


%%%%%%%%%%%%%%%%%%%%%%%%%%%%%%%%%%%%%%%%%%%%%%%%%%%%%%%%%%%%%%%%%%%%%%%%%%%%%%%%
%% FUNCTION: MANUAL PROTOCOL
%% DERSCRIPTION: Create the document subsection for a manual protocol.   The
%%    function takes 2 arguments: a basename and a protocol title.   The
%%    protocol title is used in the section header and should be distinct from
%%    all other protocol titles to ensure distinct entries in the TOC.   The
%%    basename is used to locate the protocol source files.
%%%%%%%%%%%%%%%%%%%%%%%%%%%%%%%%%%%%%%%%%%%%%%%%%%%%%%%%%%%%%%%%%%%%%%%%%%%%%%%%
\newcommand{\manualProtocol}[2]{
    \genericProtocol{#1}{#2}
        {\protocolDescription}
        {\manualTestSteps}
        {\manualTestEvidence}
}

%%%%%%%%%%%%%%%%%%%%%%%%%%%%%%%%%%%%%%%%%%%%%%%%%%%%%%%%%%%%%%%%%%%%%%%%%%%%%%%%
%% FUNCTION: automatedProtocol
%% DERSCRIPTION: Create the document subsection for an automated protocol.   The
%%    function takes 2 arguments: a basename and a protocol title.   The
%%    protocol title is used in the section header and should be distinct from
%%    all other protocol titles to ensure distinct entries in the TOC.   The
%%    basename is used to locate the protocol source files.
%%%%%%%%%%%%%%%%%%%%%%%%%%%%%%%%%%%%%%%%%%%%%%%%%%%%%%%%%%%%%%%%%%%%%%%%%%%%%%%%
\newcommand{\automatedProtocol}[2]{
    \IfFileExists{test-output/protocol/#1/recipe-results-99.tex}
       { \wlog{WARNING: Using existing results for protocol #1.}}
       { \ifdefined\runtests \runProtocol{#1};\wlog{(Re)running protocol #1};\else\wlog{WARNING: Protocol #1 needs to be run, but is disabled by user};\fi }
    \genericProtocol{#1}{#2}
        {\protocolDescription}
        {\automatedTestSteps}
        {\automatedTestEvidence}
}

%%%%%%%%%%%%%%%%%%%%%%%%%%%%%%%%%%%%%%%%%%%%%%%%%%%%%%%%%%%%%%%%%%%%%%%%%%%%%%%
%% FUNCTION: placeholder
%% DESCRIPTION: display a generic placeholder message.   This message should be
%% displayed whenever a section could not be generated because of missing
%% test results.
%%%%%%%%%%%%%%%%%%%%%%%%%%%%%%%%%%%%%%%%%%%%%%%%%%%%%%%%%%%%%%%%%%%%%%%%%%%%%%%
\newcommand{\placeholder}[0]{ \textcolor{red}{
This protocol has not been run.  This is a placeholder section that will be
replaced by the actual test results after this protocol is run.
}}
