% ------------------------------------------------------------------------------
% This set of commands define the path test-output/factory the amber driver
% produces as it process YAML files.
% ------------------------------------------------------------------------------
\newcommand{\testOutputDir}{test-output}%
\newcommand{\tofPath}{\testOutputDir}%
\newcommand{\tofPathC}{\testOutputDir}%
\newcommand{\tcStyleCaption}{\tcCsv}%

\newcommand{\setTOFPath}{%
 \renewcommand{\testOutputDir}{test-output}%
 \renewcommand{\tcStyleCaption}{\caption{\tcCsv}}%
 }%

\newcommand{\setTOFPathC}[2]{%
 \renewcommand{\testOutputDir}{test-output/#1/#2}%
 \renewcommand{\tcStyleCaption}{\caption{\tcCsvC}}%
 }%

\newcommand{\tcPath}{}%
\newcommand{\tcLog}{}%
\newcommand{\tcPng}{}%
\newcommand{\tcPngName}{}%
\newcommand{\tcCsv}{}%

\newcommand{\tcPathC}{}%
\newcommand{\tcLogC}{}%
\newcommand{\tcPngC}{}%
\newcommand{\tcCsvC}{}%
\newcounter{FileCnt}\setcounter{FileCnt}{1}%
\newboolean{MoreFiles}\setboolean{MoreFiles}{true}%

% ------------------------------------------------------------------------------
% Comma Separated Variable table definition.  This table is applicable to 1 and
% 3 argument commands.
% ------------------------------------------------------------------------------
% Define column names use by csvreader for Test Case results.
\csvnames{tcNames}{1=\id, 2=\key, 3=\des, 4=\status, 5=\imgname}

\csvstyle{tcStyle}{
   before reading=\small
  ,longtable=|L{1.6cm}|L{5.0cm}|L{6.5cm}|C{1.1cm}|
  ,table head=\hline\rowcolor{cyan!25}REQID & RESKEY & TEXT & STATUS\ER\endhead
  ,late after line=\ER
  ,table foot=\tcStyleCaption
  ,tcNames
}

% ------------------------------------------------------------------------------
% Comma Separated Variable table definition.  This table is applicable to 1 and
% 3 argument commands.
% ------------------------------------------------------------------------------
% Define column names use by csvreader for Test Case results filtered by image name.
% I want the csv reader to read in the new fifth column, and drop it.
%
\csvnames{tcNamesFiltered}{1=\id, 2=\key, 3=\des, 4=\status, 5=\imgname}

\csvstyle{tcStyleFiltered}{
	before reading=\small
	,longtable=|L{2.0cm}|L{4.5cm}|L{6.4cm}|C{1.1cm}|
	,table head=\hline\rowcolor{cyan!25}REQID & RESKEY & TEXT & STATUS\ER\endhead
	,late after line=\ER
	,table foot=\tcStyleCaption
	,tcNamesFiltered
}

% ------------------------------------------------------------------------------
% Zero fill numbers
\newcommand{\zerofill}[1]{\ifnum#1<10 00#1\else\ifnum#1<100 0#1\else #1\fi\fi}

% ------------------------------------------------------------------------------
% These commands are used when the amber driver is directed to run a Test Plan,
% Test Suite, or Test Case that does not provide support for a web browser or
% spoken language.
%
% Test Plan, Test Suite, and Test Case one argument commands:
% #1 - filename
% ------------------------------------------------------------------------------
% Define Test Plan Output (tpo) commands.
\newcommand{\tpo}[1]{\setTOFPath\inputIfExists{\tofPath/#1.tex}}

% ------------------------------------------------------------------------------
% Define Test Suite Output (tso) commands.
\newcommand{\tso}[1]{\setTOFPath\inputIfExists{\tofPath/#1.tex}}

% ------------------------------------------------------------------------------
% Extract Method to include CSV:
\newcommand{\tcIncludeCsv}{
	% ------------------------------------------------------------------------------
	% Comma Separated Variable table definition.  This table is applicable to 1 and
	% 3 argument commands.
	% ------------------------------------------------------------------------------
	% Define column names use by csvreader for Test Case results.
	\csvnames{tcNames}{1=\id, 2=\key, 3=\des, 4=\status, 5=\imgname}
	
	\csvstyle{tcStyle}{
		before reading=\small
		,longtable=|L{1.6cm}|L{5.0cm}|L{6.5cm}|C{0.9cm}|
		,table head=\hline\rowcolor{cyan!25}REQID & RESKEY & TEXT & STATE\ER\endhead
		,late after line=\ER
		,table foot=\tcStyleCaption
		,tcNames
	}%
		
	\IfFileExists{\tcCsv}{
		\csvreader[tcStyle, separator=pipe]
		{\tcCsv}{}{\id & \key & \des & \status}
	}
	{}%
}

\makeatletter
\def\instring#1#2{TT\fi\begingroup
	\edef\x{\endgroup\noexpand\in@{#1}{#2}}\x\ifin@}
\makeatother

% ------------------------------------------------------------------------------
% Extract Method to include filtered CSV:
\newcommand{\tcIncludeFilteredCsv}[1]{
	
	% ------------------------------------------------------------------------------
	% Comma Separated Variable table definition.  This table is applicable to 1 and
	% 3 argument commands.
	% ------------------------------------------------------------------------------
	% Define column names use by csvreader for Test Case results filtered by image name.
	% I want the csv reader to read in the new fifth column, and drop it.
	%
	
	\subsection{Parameter: {#1}}% for debugging purposes.
	
	\csvnames{tcNamesFiltered}{1=\id, 2=\key, 3=\des, 4=\status, 5=\imgname}
	
	\csvstyle{tcStyleFiltered}{
		before reading=\small
		,longtable=|L{2.2cm}|L{5.0cm}|L{6.4cm}|C{0.9cm}|
		,table head=\hline\rowcolor{cyan!25}REQID & RESKEY & TEXT & STATE\ER\endhead
		,late after line=\ER
		,table foot=\tcStyleCaption
		,tcNamesFiltered
	}
	
	\IfFileExists{\tcCsv}{
		%\csvreader[tcStyleFiltered, separator=pipe]
		\csvreader[tcStyleFiltered, separator=pipe, filter test=\ifcsvstrcmp{\imgname}{#1}]
		{\tcCsv}{}{\id & \key & \imgname & \status}
		%{\tcCsv}{}{\id & \key & \des & \status}
	}
	{}
	%
	
}

% ------------------------------------------------------------------------------
% Define Test Case Output (tco) commands.
%
% Define a command to standardize test case output (tco).
\newcommand{\tco}[1]{
  %
  \setTOFPath
  %
  \renewcommand{\tcPath}{\tofPath}
  \renewcommand{\tcLog}{\tcPath/#1.tex}
  \renewcommand{\tcPng}{\tcPath/#1-\zerofill{\theFileCnt}.png}
  \renewcommand{\tcPngName}{{#1}-\zerofill{\theFileCnt}.png}
  \renewcommand{\tcCsv}{\tcPath/#1.csv}
  %
  \inputIfExists{\tcLog}
  %
  % Comment out in order to apply filtered CSV files per image.
  % \tcIncludeCsv
  %
  \setboolean{MoreFiles}{true}
  \setcounter{FileCnt}{1}
  \whiledo{\boolean{MoreFiles}}{
    \IfFileExists{\tcPng}{
      \clearpage
      \begin{figure}[ht]
        \centering
        \includegraphics[scale=0.32]{\tcPng}
        \caption{#1}
      \end{figure}
      \tcIncludeFilteredCsv{\tcPngName} above uses tcPngName method. 
      \tcIncludeFilteredCsv{factory/case/bed-create/bed-create-001.png} above uses Hardcoded String name.
      \addtocounter{FileCnt}{1}
    }
    {\setboolean{MoreFiles}{false}}
  }
}
% ------------------------------------------------------------------------------
% These commands are used when the amber driver is directed to run a Test Plan,
% Test Suite, or Test Case to support a web browser and spoken language.
%
% Test Plan, Test Suite, and Test Case three argument commands:
% #1 - browser
% #2 - language
% #3 - filename
% ------------------------------------------------------------------------------
% Define Test Plan Output (tpo) commands.
\newcommand{\tpoC}[3]{\setTOFPathC{#1}{#2}\input{\tofPathC/#3.tex}}

% ------------------------------------------------------------------------------
% Define Test Suite Output (tso3) commands.
\newcommand{\tsoC}[3]{\setTOFPathC{#1}{#2}\input{\tofPathC/#3.tex}}


% ------------------------------------------------------------------------------
% Define Test Case Output (tcoC) commands.
%
% Define a command to standardize test case output (tcoC).
\newcommand{\tcoC}[3]{
  %
  \setTOFPathC{#1}{#2}
  %
  \setcounter{FileCnt}{1}
  \renewcommand{\tcPathC}{\tofPathC}
  \renewcommand{\tcLogC}{\tcPathC/#3.tex}
  \renewcommand{\tcPngC}{\tcPathC/#3-\zerofill{\theFileCnt}.png}
  \renewcommand{\tcCsvC}{\tcPathC/#3.csv}
  %
  \inputIfExists{\tcLogC}
  %
  \IfFileExists{\tcCsvC}{
    \csvreader[tcStyle, separator=pipe]
    {\tcCsvC}{}{\id & \key & \des & \status}
  }
  {}
  %
  \setboolean{MoreFiles}{true}
  \whiledo{\boolean{MoreFiles}}{
    \IfFileExists{\tcPngC}{
      \clearpage
      \begin{figure}[ht]
        \centering
        \includegraphics[scale=0.31]{\tcPngC}
        \caption{\tcPngC}
      \end{figure}
      \addtocounter{FileCnt}{1}
    }
    {\setboolean{MoreFiles}{false}}
  }
}

% ------------------------------------------------------------------------------
% Initialize our directory variables to {} so LaTeX can renew them each time
% \useAutoDocFile is referenced.
\newcommand{\lDir}{}% local directory
\newcommand{\sDir}{}% boilerplateSpecificDir
\newcommand{\bDir}{}% boilerplateDir

% Refresh directory variables. 
\newcommand{\refreshDirs}[1]{
  \renewcommand{\lDir}{#1}%
  \renewcommand{\sDir}{\boilerplateSpecificTestDir/#1}%
  \renewcommand{\bDir}{\boilerplateDir/#1}%
}

% We want to find and use a document part by searching the local directory,
% followed by the boilerPlateSpecificDir, and finally the boilerplateDir.
\newcommand{\useAutoDocFile}[1]{%
  \refreshDirs{#1}
  \IfFileExists{\lDir}{\input{\lDir}}{%
    \IfFileExists{\sDir}{\input{\sDir}}{%
      \IfFileExists{\bDir}{\input{\bDir}}{}%
    }%
  }%
}%

% ------------------------------------------------------------------------------
