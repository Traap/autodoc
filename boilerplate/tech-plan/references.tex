\section{Software Product References}
\begin{longtable}[ht]{|L{3.5cm}|L{5.5cm}|L{5.5cm}|}\hline%
  \rowcolor{fkblue}%
  \fkHdrRow{Item} & \fkHdrRow{Number} & \fkHdrRow{Title/Name}\ER%
  \endhead%
  SOFTWARE DEVELOPMENT PROCESS & SOP-PRC02004[G] & Software Development Process \ER%
    CHANGE CONTROL REFERENCE & \input{data/dnd-plan-number.tex} & \input{data/dnd-plan-title.tex} \ER% 
    Document Control System Application & N/A & SmartSolve 9.0 \ER%
    PROJECT DESIGN HISTORY FILE & \input{data/pkg-num-project.tex} & \input{data/pkg-title-project} \ER%
    SOFTWARE DESIGN HISTORY FILE & \input{data/pkg-num-sw.tex} & \input{data/pkg-title-sw} \ER%
\caption{Software Product References}
\label{table:1}
\end{longtable}%


The software product will be developed per the SOFTWARE DEVELOPMENT PROCESS standard operating procedure. The standard operating procedure defines the processes employed, activities and tasks to be performed, traceability to be established, and deliverables to be completed for the software product.

The software product is the VSS project, which is defined by the CHANGE CONTROL REFERENCE. The Design History File for the project is maintained in the Fenwal Document Control System under the PROJECT DESIGN HISTORY FILE. For organizational convenience a software product specific SOFTWARE DESIGN HISTORY FILE is also maintained within the Fenwal Document Control System, and references PROJECT DESIGN HISTORY FILE as its master document. All deliverables for the software product managed through the Fenwal Document Control System will directly or indirectly reference the SOFTWARE DESIGN HISTORY FILE. As appropriate, additional packages may be defined for the software product. Such packages will also directly or indirectly reference the SOFTWARE DESIGN HISTORY FILE.
