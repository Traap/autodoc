\newpage
\section{Test Plan Overview}
This section describes Test Plans, Test Suites, Test Cases, and Test Steps that
demonstrate how a Configuration Item satisfies the IUR.  Each Test Plan
describes any setup criteria needed to conduct the Test Steps. Each Test Plan
contains a list of IUR\textbackslash{s} and the Test Steps that demonstrate how
the Configuration Item satisfies the IUR. Each Test Step is marked passed or
failed as it is completed.  Each Test Plan is marked passed when all Test Steps
pass or failed if a single Test Step fails.  This serves as a record of
completed Test Plans and Test Steps.

Each Test Plan is described in its own section.  The order the Test Plans
are listed is the order they are run.  Each Test Plan defines:

\begin{description}[labelindent=25pt, style=multiline, leftmargin=4.0cm]

\item[name]
  Each Plan, Suite, and Case has a unique name.

\item[purpose]
  Each Plan, Suite, and Case has a purpose.

\item[Test Steps]
  Each step has a confirmation and expectation along with the command needed to
  challenge the IUR.

\item[Objective Evidence]
  A record the Test Plan was run along with any evidence collected while the
  Test Steps were run.

\item[Traceability]
  Suites and Cases are traced to an IUR that is challenged. IUR can be traced to
  multiple Suites and Cases.

\end{description}

Each Test Plan, Test Suite, Test Case,  and Test Step has been designed to be
run by the computer. However, a person may choose to manually run the Test
Plans, save the test results, and generate this test report as specified in the
appropriate design documentation.

The example below runs two commands: 1) git help and 2) cat ~/gitconfig.  The
output from both commands are written to the system console.

\begin{lstlisting}
plan:
  name: A Test Plan Name
  purpose: purpose of the plan

suite:
  name: A Test Suite Name
  purpose: a suite purpose
  requirement: IUR01 and IUR02

  - case:
    name: A Test Case name
    purpose: A Test Case purpose
    steps:
      - confirm: Confirm git help is written to the console output.
        expectation: Git help is displayed.
        command: git
        argument: help

      - confirm: Confirm .git config is written to the console.
        expectation: .gitconfig is written to the console output.
        command: cat
        argument: .gitconfig
\end{lstlisting}
