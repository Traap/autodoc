\newpage
\subsection{Protocol - Tag Commits}
This protocol is used to demontrate Git's abbility to tag a git commit.  The Git
repository, fhsws-git-validation, has been pre-seeded with the following tags:

\begin{longtable}
{|C{1.2cm}|L{10cm}|} \hline
Tag & Git Checksum\\ \hline
v1.0 & bcde6bf1ed2841f39e719ccd323df4cc33497e04 \\ \hline
v1.1 & b64d30b59ff026f12350a5c040d779a5a9a59214 \\ \hline
v1.2 & 8aa0308a4b661e97df4002ccd7f0ee3dd2f55994 \\ \hline
v2.0 & ea06210d6f84c2daf26f00c39643a9cd9b98a4a5 \\ \hline
v2.1 & 440ba8b56feb5941783b9cb57ed192391cd251c6 \\ \hline
v2.2 & c47279b87ed77413792908818215f53364567bb1 \\ \hline
\end{longtable}

The Git Checksum is unique to a Git installation and are listed here as
examples.  In the event these tags do not exist, the following commands are used
to recreate them and publish them to the git server: \\

\begin{verbatim}
       git tag v1.0 [git-checksum]

            where git-checksum is the git-commit-id you are associating
            the tags with.  Git-commit-id's are globally-uniquie.

        git push origin --tags
\end{verbatim}

\newpage
\subsubsection{Test Steps}
\IfFileExists{/tmp/test-output/git/protocol/TagCommits/recipe.tex}
      {This protocol has not been run.  This is a placeholder section that will be
replaced by the actual test steps after this protocol is run.}
      {This protocol has not been run.  This is a placeholder section that will be
replaced by the actual test steps after this protocol is run.}

\newpage
\subsubsection{Test Script}
\lstinputlisting[language=Haskell]{src/TagCommits/Protocol.hs}

\newpage
\subsubsection{Test Evidence}
\IfFileExists{/tmp/test-output/git/protocol/TagCommits/results.tex}
      {The Test Plan, Test Suite, Test Case, and Test Steps related to this report have
not been run yet.  This is a placeholder section that will be replaced with the
actual test results when the automated tests are run.
}
      {The Test Plan, Test Suite, Test Case, and Test Steps related to this report have
not been run yet.  This is a placeholder section that will be replaced with the
actual test results when the automated tests are run.
}
