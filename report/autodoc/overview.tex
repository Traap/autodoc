\section{Overview}
Starterkit is a document that is used to demonstrate how \skAutoDoc can be
used to setup a unit test, verification test, a tool validation test
document, or a test record document.

An \skAutoDoc report consists of two distinct parts: 1) text and 2) a
program. Text describes requirements, test expectation, and test results, and
a program automates running Test Plans, Test Suites, and Test Steps that are
compliant with the \skOutputFactory specified by the \skAmber driver.

The goal of this report is to demonstrate how to specialize \skAutoDoc for your
specific purposes.  This report does not hook into \skAmber.  \skAmber has
a similar report that describes the requirements for the \skInputFactory and
the \skOutputFactory.

Several sections in section 1 will seam out of place, and perhaps will read
akward. I have included them so that you will see \skAutoDoc framework reuses
documentation parts.

\subsection{\skStarterkit}
The contents of \skStarterkit are listed below.  Page layout, headers, footers,
and table of contents is driven by the \LaTeX\ document class \skTlcArticle.

\begin{verbatim}
  \documentclass[10pt]{tlc-article}
  \begin{document}
    \tlcTitlePageAndTableOfContents
      {Autodoc \\ Starter Kit}
      {Gary A. Howard}
      {This document demonstrates how-to use (\skAutoDoc) a \LaTeX\
      documentation framework. \skAutoDoc declares this a tool-test document
      type.}

    \def\boilerplateSpecificDir{\designReviewDir}
\useAutoDocFile{document-outline.tex}


    \makeatletter
    \tlcDebug
    \autodocDebug
    \makeatother

  \end{document}
\end{verbatim}

\subsection{\skToc}
\textbf{\skTlcArticle} provides the \skToc to provide the document title, table
of contents, author name, and document abstract that you read on the first page.

\subsection{additional-layout}
\skTlcArticle provides a hook to \skAdditionalLayout to allow you to specialize
your documents layout.  The primary contents of this documents
\skAdditionalLayout file is shown below.

\begin{verbatim}
  \def\autodocDir{..}
  % ------------------------------------------------------------------------------
% We create definitions to support various relative paths used.
\def\boilerplateDir{\autodocDir/boilerplate}

% ------------------------------------------------------------------------------
% Boilerplate directories. 

\def\debugTestDir{\boilerplateDir/debug-test}
\def\placeholderDir{\boilerplateDir/placeholder}
\def\stdPhrasesDir{\boilerplateDir/standard-phrases}

% ------------------------------------------------------------------------------
% Specific document types.

\def\techPlanDir{\boilerplateDir/tech-plan}
\def\testPlanDir{\boilerplateDir/test-plan}
\def\testRecordDir{\boilerplateDir/test-record}
\def\testReportDir{\boilerplateDir/test-report}
\def\toolTestDir{\boilerplateDir/tool-test}
\def\unitTestDir{\boilerplateDir/unit-test}
\def\verTestDir{\boilerplateDir/ver-test}

% ------------------------------------------------------------------------------
% Autodoc data and shared folders.
\global\def\dataDir{\autodocDir/data}
\global\def\sharedDir{\autodocDir/shared}

% ------------------------------------------------------------------------------
% Include macros, debugging aids, SOPs, and standard phrases common to all
% documents.
% ------------------------------------------------------------------------------
% This set of commands define the path test-output/factory the amber driver
% produces as it process YAML files.
% ------------------------------------------------------------------------------
\newcommand{\tcLog}{}%
\newcommand{\tcPng}{}%
\newcommand{\tcCsv}{}%
\newcommand{\tcCap}{}%
\newcommand{\tcStyleCaption}{\tcCsv}%

\newboolean{MoreSteps}\setboolean{MoreSteps}{true}%
\newcounter{StepCnt}\setcounter{StepCnt}{1}%

\newboolean{MoreFiles}\setboolean{MoreFiles}{true}%
\newcounter{FileCnt}\setcounter{FileCnt}{1}%


\newcommand{\setTOFPath}[1]{%
 \renewcommand{\tcCsv}{#1.csv}%
 \renewcommand{\tcLog}{#1.tex}%
 \renewcommand{\tcPng}{#1-\zerofill{\theStepCnt}-\zerofill{\theFileCnt}.png}
 \renewcommand{\tcCap}{\tcPng}
 \renewcommand{\tcStyleCaption}{\caption{\tcCsv}}%
}%

\newcommand{\setTOFPathC}[3]{%
 \renewcommand{\tcCsv}{#3.csv}%
 \renewcommand{\tcLog}{#3.tex}%
 \renewcommand{\tcPng}{#3-\zerofill{\theStepCnt}-\zerofill{\theFileCnt}.png}
 \renewcommand{\tcCap}{\tcPng}
 \renewcommand{\tcStyleCaption}{\caption{\tcCsv}}%
}%


% ------------------------------------------------------------------------------
% Comma Separated Variable table definition.  This table is applicable to 1 and
% 3 argument commands.
% ------------------------------------------------------------------------------
% Define column names use by csvreader for Test Case results.
\csvnames{tcNames}{1=\id, 2=\key, 3=\des, 4=\status}%

\csvstyle{tcStyle}{%
   before reading=\small
  ,longtable=|L{1.6cm}|L{5.0cm}|L{6.5cm}|C{1.1cm}|
  ,table head=\hline\rowcolor{cyan!25}REQID & RESKEY & TEXT & STATUS\ER\endhead
  ,late after line=\ER
  ,table foot=\tcStyleCaption
  ,tcNames
}%

% ------------------------------------------------------------------------------
% Zero fill numbers
\newcommand{\zerofill}[1]{\ifnum#1<10 00#1\else\ifnum#1<100 0#1\else #1\fi\fi}%

% ------------------------------------------------------------------------------
% These commands are used when the amber driver is directed to run a Test Plan,
% Test Suite, or Test Case that does not provide support for a web browser or
% spoken language.
%
% Test Plan, Test Suite, and Test Case one argument commands:
% #1 - filename
% ------------------------------------------------------------------------------
% Define Test Plan Output (tpo) commands.
\newcommand{\tpo}[1]{\setTOFPath{#1}\inputIfExists{#1.tex}}%

% ------------------------------------------------------------------------------
% Define Test Suite Output (tso) commands.
\newcommand{\tso}[1]{\setTOFPath{#1}\inputIfExists{#1.tex}}%

% ------------------------------------------------------------------------------
% Define Test Case Output (tco) commands.
%
% Define a command to standardize test case output (tco).
\newcommand{\tco}[1]{\setTOFPath{#1}\assembleTestCaseOutput}%

% ------------------------------------------------------------------------------
% These commands are used when the amber driver is directed to run a Test Plan,
% Test Suite, or Test Case to support a web browser and spoken language.
%
% Test Plan, Test Suite, and Test Case three argument commands:
% #1 - browser
% #2 - language
% #3 - filename
% ------------------------------------------------------------------------------
% Define Test Plan Output (tpo) commands.
\newcommand{\tpoC}[3]{\setTOFPath{#1}{#2}{#3}\inputIfExists{#3.tex}}%

% ------------------------------------------------------------------------------
% Define Test Suite Output (tso3) commands.
\newcommand{\tsoC}[3]{\setTOFPath{#1}{#2}{#3}\inputIfExists{#3.tex}}%

% ------------------------------------------------------------------------------
% Define Test Case Output (tcoC) commands.
%
% Define a command to standardize test case output (tcoC).
\newcommand{\tcoC}[3]{\setTOFPathC{#1}{#2}{#3}\assembleTestCaseOutput}%

% ------------------------------------------------------------------------------
% Assemble test case output.
% includeLogFile
% includeCsvFile
% includeGraphicFiles
% ------------------------------------------------------------------------------
\newcommand{\assembleTestCaseOutput}{%
  %
  % Include the log file Amber created.
  %
  \inputIfExists{\tcLog}%
  %
  % Include the CSV file the program Amber called created.
  %
  \IfFileExists{\tcCsv}{%
    \csvreader[tcStyle, separator=pipe]{\tcCsv}{}{\id & \key & \des & \status}%
  }{}%
  %
  % Include any graphic file the program Amber called created.
  %
  \setboolean{MoreSteps}{true}%
  \setcounter{StepCnt}{1}%

  \setboolean{MoreFiles}{true}%
  \setcounter{FileCnt}{1}%
  %
  \whiledo{\boolean{MoreSteps}}{%
    \setboolean{MoreFiles}{true}%
    \IfFileExists{\tcPng}{%
      \whiledo{\boolean{MoreFiles}}{%
        \IfFileExists{\tcPng}{%
          \clearpage%
          \begin{figure}[ht]%
            \centering%
            \includegraphics[scale=0.31]{\tcPng}%
            \caption{\footnotesize{\tcCap}}%
          \end{figure}%
          \addtocounter{FileCnt}{1}%
        }%
        {% Else
          \setboolean{MoreFiles}{false}%
          \setcounter{FileCnt}{1}%
        }%
      }%
      \addtocounter{StepCnt}{1}%
    }%
    {% Else
      \setboolean{MoreSteps}{false}%
      \setcounter{StepCnt}{1}%
    }%
  }%
}%

% ------------------------------------------------------------------------------
% Initialize our directory variables to {} so LaTeX can renew them each time
% \useAutoDocFile is referenced.
\newcommand{\lDir}{}% local directory
\newcommand{\sDir}{}% boilerplateSpecificDir
\newcommand{\bDir}{}% boilerplateDir

% Refresh directory variables. 
\newcommand{\refreshDirs}[1]{%
  \renewcommand{\lDir}{#1}%
  \renewcommand{\sDir}{\boilerplateSpecificTestDir/#1}%
  \renewcommand{\bDir}{\boilerplateDir/#1}%
}

% We want to find and use a document part by searching the local directory,
% followed by the boilerPlateSpecificDir, and finally the boilerplateDir.
\newcommand{\useAutoDocFile}[1]{%
  \refreshDirs{#1}
  \IfFileExists{\lDir}{\input{\lDir}}{%
    \IfFileExists{\sDir}{\input{\sDir}}{%
      \IfFileExists{\bDir}{\input{\bDir}}{}%
    }%
  }%
}%

% ------------------------------------------------------------------------------

% ------------------------------------------------------------------------------
% We define autodocDebug to aid the user when troubleshooting relative paths.
% This macro should be placed at the end of our main document so that you can
% debug text expansion LaTeX compiler is performing.:w
\newcommand{\autodocDebug}{%
  \clearpage
  \section{autodoc debug}
  \subsection{autodoc definitions}
    \begin{description}[align=right,leftmargin=*,labelindent=5cm]
      \item[autodocDir:] \autodocDir
      \item[boilerplateDir:] \boilerplateDir
      \item[dataDir:] \dataDir
      \item[placeholderDir:] \placeholderDir
      \item[sharedDir:] \sharedDir
      \item[testRecordDir:] \testRecordDir
      \item[toolTestDir:] \toolTestDir
      \item[unitTestDir:] \unitTestDir
      \item[verTestDir:] \verTestDir
      \item[boilerplateSpecificTestDir:] \boilerplateSpecificTestDir
  \end{description}
}%
% ------------------------------------------------------------------------------

% ------------------------------------------------------------------------------
% The boilerplate has hooks for two SOPs. 
\def\sopSDLC{SOP-PRC02004 Software Development Procedure}
\def\sopSCM{SOP-PRC02004 Software Development Procedure; Software
Configuration Management}

% ------------------------------------------------------------------------------

% ------------------------------------------------------------------------------
% Standard Code Review Abstract.
% ------------------------------------------------------------------------------
\newcommand{\stdCodeReviewType}{Code Review}
\newcommand{\stdCodeReview}[1]{#1 \stdCodeReviewType}

\newcommand{\stdCodeReviewAbstract}[1]{%
  This is a #1 report.  The #1 is stored in \Gls{company} \stdQMS\
  after observations have been dispositioned.
}

% ------------------------------------------------------------------------------
% Standard Code Review document generator command.
% ------------------------------------------------------------------------------
\newcommand{\stdCodeReviewDoc}[1]{%
  \stdAutoDocLayout{\stdCodeReview{#1}}%
                   {\stdCodeReviewAbstract{#1}}%
                   {\codeReviewDir/main.tex}%
}%

% ------------------------------------------------------------------------------


% ------------------------------------------------------------------------------

\end{verbatim}

\skAutoDocDir is the location on your file system of the \skAutoDoc repository.
\skProjectDirTex is an \skAutoDoc file that defines locations of files within
the \skAutoDoc folder.

\subsection{main}
All text files found in \skBoilerplate are common to all \skAutoDoc documents.
Each document consist of four major sections:

\begin{description}[leftmargin=3cm, style=nextline]
  \item[Overview:] This section provide an overview of the document.  It has
    been designed with quality and regulatory compliance in mind.  The goal is
    to assemble documents that a suitable for your Design History File, or may
    be used when submitting records to the Food and Drug Administration.

  \item[Requirements:] Identifies the requirements of the document.  This may be
    user needs, products requirements, or testing requirements.

  \item[Test Plan Overview:] This section will vary based on the type of testing
    that is being described.  This section will contain test evidence when
    combined with \skAmber.

  \item[Conclusion:] This section is the conclusion you write after you have
    reviewed your test evidence.  By default, \skAutoDoc, provides a default
    conclusion that you override by placing a file named \skConclusion in same
    directory as \skStarterkit.

\end{description}

\skAutoDoc provides three document templates.

\begin{description}[leftmargin=3cm, style=nextline]
  \item[\skToolTest:] This template was designed to standardize
    \skToolTestTemplate.

  \item[\skUnitTest:] This template can be used when your project requires
    \skUnitTestTemplate.

  \item[\skVerTest:] \skVerTestTemplate template is use do when you need to
    record \skUnitTestTemplate results.

\end{description}

The term Test has been abstracted to mean any test type including: unit test,
verification test, validation test, integration test and tool validation test.

\subsection{\skTlcDebug}
\skTlcDebug is a command provided by \skTlcArticle to assist you when you are
having difficulty with \LaTeX\ text expansion.  You will notice the final
sections of this document has debugging information.

\subsection{\skAutoDocDebug}
\skAutoDocDebug is similar to \skTlcDebug with a focus on debugging \skAutoDoc
documents.
