% {{{ Folder Structure

\clearpage
\section{Folder Structure} 
This section provides a high-level overview for the autodoc folder structure.
Strict naming conventions are used throughout Autodoc.  Autodoc has
a boilerplate and report section, where boilerplate is used to provide standard
\LaTeX\ definitions, and report is used to demonstrate the boilerplate templates
function correctly. 

\tlcVspace

\subsection{autodoc}
\dirtree{%
.1 autodoc.
.2 boilerplate.
.2 report.
}%

\subsection{boilerplate}
Subsequent sections expand each boilerplate folder.  A Boilerplate is reusable
documentation parts like definitions, paragraphs, document layout, terms, and
phrases, and commands. 

\tlcVspace

\dirtree{%
.1 autodoc.
.2 boilerplate.
.3 data.
.3 code-review.
.3 design-review.
.3 macros.
.3 placeholder.
.3 standard-phrases.
.3 test-plan.
.3 tech-plan.
.3 test-record.
.3 test-report.
.3 tool-test.
.3 unit-test.
.3 ver-test.
.2 report.
}%

\subsection{report}
Subsequent sections also expand each report folder.  A report is a concrete
implementation of a boilerplate item.  This document shows the content of each
file referenced.  Whereas building a autodoc report reveals how the boilerplate
information is typeset. 

\tlcVspace

\dirtree{%
.1 autodoc.
.2 boilerplate.
.2 report.
.3 code-review.
.3 design-review.
.3 glossaries.
.3 shared.
.3 test-plan.
.3 tech-plan.
.3 test-record.
.3 test-report.
.3 tool-test.
.3 unit-test.
.3 ver-test.
}

\clearpage
\subsection{data}
tlc-article.cls looks for a document/data/additional-layout.tex file.  When
found, the content is consumed.  Autodoc demonstrates two important concepts in
this section.  All document types benefit from autodoc definitions defined in
boilerplate/date folder.  The share folder is used to demonstrate how a project
can hook into autodoc/boilerplate/data folder and then override or add
additional definitions. 

\tlcVspace

\dirtree{%
.1 autodoc.
.2 boilerplate.
.3 data.
.4 abbreviations.
.4 additional-layout.tex.
.4 definitions.tex.
.2 report.
.3 shared.
.4 data.
.5 additional-layout.tex.
.5 header-footer.tex.
}%

% -------------------------------------------------------------------------- }}}
