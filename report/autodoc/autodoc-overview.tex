% {{{ Autodoc Overview 

\section{Autodoc Overview}

\todo[inline]{Complete section. This section has content that needs to move to
other sections.}%

The contents of \adStarterkit\ are listed below.  Page layout, headers, footers,
and table of contents is driven by the \LaTeX\ document class \adTlcArticle.

\begin{verbatim}
  \documentclass[10pt]{tlc-article}
  \begin{document}
    \tlcTitlePageAndTableOfContents
      {Autodoc \\ Starter Kit}
      {Jane Doe}
      {This document demonstrates how-to use (\adAutoDoc) a \LaTeX\
      documentation framework. \adAutoDoc declares this a tool-test document
      type.}

    \def\boilerplateSpecificTestDir{\toolTestDir}
\useAutoDocFile{document-outline.tex}


    \makeatletter
    \tlcDebug
    \autodocDebug
    \makeatother

  \end{document}
\end{verbatim}

\subsection{\adToc}
\textbf{\adTlcArticle} provides the \adToc\ to provide the document title, table
of contents, author name, and document abstract that you read on the first page.

\subsection{additional-layout}
\adTlcArticle\  provides a hook to \adAdditionalLayout\ to allow you to specialize
your documents layout.  The primary contents of this documents
\adAdditionalLayout\ file is shown below.

\begin{verbatim}
  \def\autodocDir{..}
  % ------------------------------------------------------------------------------
% We create definitions to support various relative paths used.
\def\boilerplateDir{\autodocDir/boilerplate}

% ------------------------------------------------------------------------------
% Boilerplate directories. 

\def\debugTestDir{\boilerplateDir/debug-test}
\def\placeholderDir{\boilerplateDir/placeholder}
\def\stdPhrasesDir{\boilerplateDir/standard-phrases}

% ------------------------------------------------------------------------------
% Specific document types.

\def\designReviewDir{\boilerplateDir/design-review}
\def\techPlanDir{\boilerplateDir/tech-plan}
\def\testPlanDir{\boilerplateDir/test-plan}
\def\testRecordDir{\boilerplateDir/test-record}
\def\testReportDir{\boilerplateDir/test-report}
\def\toolTestDir{\boilerplateDir/tool-test}
\def\unitTestDir{\boilerplateDir/unit-test}
\def\verTestDir{\boilerplateDir/ver-test}

% ------------------------------------------------------------------------------
% Autodoc data and shared folders.
\global\def\dataDir{\autodocDir/data}
\global\def\sharedDir{\autodocDir/shared}

% ------------------------------------------------------------------------------
% Include macros, debugging aids, SOPs, and standard phrases common to all
% documents.
% ------------------------------------------------------------------------------
% Generic macros useful to any document using the autodoc framework. 
% ------------------------------------------------------------------------------
% These API's return fully-formatted tables containing the queried information
% ------------------------------------------------------------------------------
% Design History File index section.
% ------------------------------------------------------------------------------
% Define column names used by csvreader for the design history file.
\csvnames{dhfNames}%
{1=\pj,2=\pv,3=\ty,4=\dn,5=\ver,6=\ttl,7=\pkg,8=\dra,9=\drb,10=\drc,11=\drd,12=\dre,13=\ai}%

% Define column names used by csvreader for the SOP design history file.
\csvnames{dhfSopNames}%
{1=\title,2=\number,3=\version}%

% Define a table style for the design history file csvreader.
\csvstyle{dhfStyle}{%
  longtable=|C{0.5cm}|L{3cm}|C{1.5cm}|L{9cm}|
  ,table head=\hline \rowcolor{fkblue}\# & \fkHdrRow{Document} & V\fkHdrRow{Version} & \fkHdrRow{Title} \\ \hline \hline \endhead
  ,late after line=\\\hline
  ,dhfNames
}

% ------------------------------------------------------------------------------
% Define commands to select rows from the design history file.
% Parameters:
% #1 - relative path to design history file.
% #2 - project column value 
% #3 - project version column value
% #4 - document type or AI column value
\newcommand{\dhfList}[4]{\dhfQuery{#1}{\pj}{#2}{\pv}{#3}{\ty}{#4}}%
\newcommand{\aiList}[4] {\dhfQuery{#1}{\pj}{#2}{\pv}{#3}{\ai}{#4}}%

% ------------------------------------------------------------------------------
% Define a command to select a row from the SOP design history file.
% Parameters:
% #1 - relative path to the SOP design history file.
% #2 - title value
\newcommand{\dhfSopList}[2]{\dhfSopQuery{#1}{\title}{#2}}%

% -----------------------------------------------------------------------------
% Define a command to query a row from the design history SOP file.
% Parameters:
% #1 - relative path to design history SOP file.
%
% #2 - first column name to match (\title)
% #3 - first column value to match
\newcommand{\dhfSopQuery}[3]{%
  \IfFileExists{#1}{%
      \csvreader[separator=pipe, respect all,%
      filter={\equal{#2}{#3}}]%
  {#1}{}{\thecsvrow&\number[\version]}%
  }%
  {\todo[inline]{#1 was not found.}%
  }%
}%

% -----------------------------------------------------------------------------
% Define a command to query rows from the design history file.
% Parameters:
% #1 - relative path to design history file.
%
% #2 - first column name to match
% #3 - first column value to match 
%
% #4 - second column name to match
% #5 - second column value to match 
%
% #6 - third column name to match
% #7 - third column value to match 
\newcommand{\dhfQuery}[7]{%
  \IfFileExists{#1}{%
    \begin{center}%
      \csvreader[dhfStyle, separator=pipe, respect all,%
        filter={\equal{#2}{#3} \and \equal{#4}{#5} \and \equal{#6}{#7}}%
        ]%
        {#1}{}{\thecsvrow&\dn&\ver&\ttl}%
    \end{center}%
  }%
  {\todo[inline]{#1 was not found.}%
  }%
}%

% ------------------------------------------------------------------------------

% ------------------------------------------------------------------------------
% Define a command to provide a standard layout for a graphical file. 
% Parameters
%   1 - relative path to graphic file.
%   2 - figure capture 
% ------------------------------------------------------------------------------
\newcommand{\autodocGraphic}[2]{%
  \IfFileExists{#1}{%
    \begin{figure}[ht]%
      \centering%
      \includegraphics[width=\textwidth]{#1}%
      \caption{#2}%
    \end{figure}%
  }{%Else
     \todo[inline]{#1 was not found!}%
  }%
}%
% ------------------------------------------------------------------------------
% ------------------------------------------------------------------------------
% Define a standard tabular layout for documents referenced through design
% control documentation.
% ------------------------------------------------------------------------------
% Define column names use by csvreader for refdocFile.
\csvnames{refdocNames}{1=\docTitle,2=\docNumber}

% Define a table style for the refdoc table section.
\csvstyle{refdocStyle}{%
   longtable=|C{0.5cm}|L{10.5cm}|L{4.0cm}|
   ,table head=\hline \rowcolor{cyan}\# & Document Title & Document Number\\\hline\hline\endhead
  ,late after line=\\\hline
  ,refdocNames
}%

% Define a command to list documents referenced.
% Parameters
% #1 - reference document file name 
\newcommand{\refdocList}[1]{%
  \IfFileExists{#1}{%
    \begin{center}%
      \csvreader[refdocStyle,separator=pipe]%
        {\refdocFile}{}{\thecsvrow&\docTitle&\docNumber}%
    \end{center}%
  }%
  {\todo[inline]{#1 was not found.}%
  }%
}%
% ------------------------------------------------------------------------------

% ------------------------------------------------------------------------------
% Static analysis section.
% ------------------------------------------------------------------------------
% Define column names use by csvreader for static analysis File.
\csvnames{staticAnalysisNames}{1=\rule, 2=\description}

% Define a table style for the static analysis table reader.
\csvstyle{staticAnalysisStyle}{%
  longtable=|C{1cm}|L{2cm}|L{11.5cm}|
  ,table head=\hline \rowcolor{cyan}\# & Rule & Description \\ \hline \hline \endhead
  ,late after line=\\\hline
  ,staticAnalysisNames
}

% Define a command to select rows from the static analysis file.
% Parameter:
% #1 - relative path to static analysis file.
\newcommand{\staticAnalysisList}[1]{%
  \IfFileExists{#1}{%
    \begin{center}%
      \csvreader[staticAnalysisStyle, separator=pipe]%
        {#1}{}{\thecsvrow&\rule&\description}%
    \end{center}%
  }%
  {\todo[inline]{#1 was not found.}%
  }%
}%
% ------------------------------------------------------------------------------

\clearpage
\section{Test Evidence}
The Company's automation framework assembles the content in this section.  The
secion has one or more Test Plans, Test Suites, Test Cases, and Test Evidence.
The evidence provided is used to conclude the Tool has meet the Intended Use
Requirements.


% ------------------------------------------------------------------------------
% Initialize our directory variables to {} so LaTeX can renew them each time
% \useAutoDocFile is referenced.
\newcommand{\lDir}{}% local directory
\newcommand{\sDir}{}% boilerplateSpecificDir
\newcommand{\bDir}{}% boilerplateDir

% Refresh directory variables. 
\newcommand{\refreshDirs}[1]{%
  \renewcommand{\lDir}{#1}%
  \renewcommand{\sDir}{\boilerplateSpecificDir/#1}%
  \renewcommand{\bDir}{\boilerplateDir/#1}%
}

% We want to find and use a document part by searching the local directory,
% followed by the boilerPlateSpecificDir, and finally the boilerplateDir.
\newcommand{\useAutoDocFile}[1]{%
  \refreshDirs{#1}
  \IfFileExists{\lDir}{\input{\lDir}}{%
    \IfFileExists{\sDir}{\input{\sDir}}{%
      \IfFileExists{\bDir}{\input{\bDir}}{}%
    }%
  }%
}%
% ------------------------------------------------------------------------------

% -----------------------------------------------------------------------------
% Dark blue color defined at
% http://cintasartwork.millmats.com/files14/1131800/m-01306914.pdf
% -----------------------------------------------------------------------------

\definecolor{fkblue}{RGB}{0,99,190}

% ------------------------------------------------------------------------------

% {{{ command: autodocDebug
%
% We define autodocDebug to aid the user when troubleshooting relative paths.
% This macro should be placed at the end of our main document so that you can
% debug text expansion LaTeX compiler is performing.

\newcommand{\autodocDebug}{%
  \tlcDebug%
  \makeatother%
  \subsection{autodoc definitions}%
    \begin{description}[align=right,leftmargin=*,labelindent=5cm]%
      \item[autodocDir:]     \autodocDir%
      \item[boilerplateDir:] \boilerplateDir%
      \item[dataDir:]        \dataDir%
      \item[designReviewDir:]\designReviewDir%
      \item[placeholderDir:] \placeholderDir%
      \item[sharedDir:]      \sharedDir%
      \item[stdPhrasesDir:]  \stdPhrasesDir%
      \item[techPlanDir:]    \techPlanDir%
      \item[testPlanDir:]    \testPlanDir%
      \item[testRecordDir:]  \testRecordDir%
      \item[testReportDir:]  \testReportDir%
      \item[toolTestDir:]    \toolTestDir%
      \item[unitTestDir:]    \unitTestDir%
      \item[verTestDir:]     \verTestDir%
      \item[boilerplateSpecificDir:] \boilerplateSpecificDir%
    \end{description}%
}%

% -------------------------------------------------------------------------- }}}

% {{{ Audodoc references the following Standard Operating Procedures (SOP).

\def\sopGDP{SOP001 Good Documentation Practices}

\def\sopISS{SOP002 Change Management}

\def\sopSCM{SOP003 Software Configuration Management}

\def\sopSDLC{SOP004 Software Devopment Procedure}

\def\sopPAP{SOP005 Validation of Computerized Systems}

% -------------------------------------------------------------------------- }}}

% ------------------------------------------------------------------------------
% Standard Unit Test Abstract.
% ------------------------------------------------------------------------------
\newcommand{\stdUnitTypeTest}{Unit Test}
\newcommand{\stdUnitTest}[1]{#1 \stdUnitTypeTest}

\newcommand{\stdUnitTestAbstract}[1]{%
 #1 is an item in \Gls{dhf}, and stored in \Gls{company} \Gls{dms} after #1 has
 been verified.
}

% ------------------------------------------------------------------------------
% Standard Unit Test document generator command.
% ------------------------------------------------------------------------------
\newcommand{\stdUnitTestDoc}[1]{%
  \stdAutoDocLayout{\stdUnitTest{#1}}%
                   {\stdUnitTestAbstract{#1}}%
                   {\unitTestDir/main.tex}%
}%

% ------------------------------------------------------------------------------


% ------------------------------------------------------------------------------

\end{verbatim}

\adAutoDocDir\ is the location on your file system of the \adAutoDoc\ repository.
\adProjectDirTex\ is an \adAutoDoc\ file that defines locations of files within
the \adAutoDoc\ folder.

\subsection{main}
All text files found in \adBoilerplate\ are common to all \adAutoDoc\ documents.
Each document consist of four major sections:

\begin{description}[leftmargin=3cm, style=nextline]
  \item[Overview:] This section provide an overview of the document.  It has
    been designed with quality and regulatory compliance in mind.  The goal is
    to assemble documents that a suitable for your Design History File, or may
    be used when submitting records to the Food and Drug Administration.

  \item[Requirements:] Identifies the requirements of the document.  This may be
    user needs, products requirements, or testing requirements.

  \item[Test Plan Overview:] This section will vary based on the type of testing
    that is being described.  This section will contain test evidence when
    combined with \adAmber.

  \item[Conclusion:] This section is the conclusion you write after you have
    reviewed your test evidence.  By default, \adAutoDoc, provides a default
    conclusion that you override by placing a file named \adConclusion\ in same
    directory as \adStarterkit.

\end{description}

\adAutoDoc\ provides three document templates.

\begin{description}[leftmargin=3cm, style=nextline]
  \item[\adToolTest:] This template was designed to standardize
    \adToolTestTemplate.

  \item[\adUnitTest:] This template can be used when your project requires
    \adUnitTestTemplate.

  \item[\adVerTest:] \adVerTestTemplate\ template is use do when you need to
    record \adUnitTestTemplate\ results.

\end{description}

The term Test has been abstracted to mean any test type including: unit test,
verification test, validation test, integration test and tool validation test.

\subsection{\adTlcDebug}
\adTlcDebug\ is a command provided by \adTlcArticle\ to assist you when you are
having difficulty with \LaTeX\ text expansion.  You will notice the final
sections of this document has debugging information.

\subsection{\adAutoDocDebug}
\adAutoDocDebug\ is similar to \adTlcDebug\ with a focus on debugging \adAutoDoc\
documents.

% -------------------------------------------------------------------------- }}}
