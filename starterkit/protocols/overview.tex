\section{Overview}
Starterkit is a tool validation sample program.

A tool validation package consists of two distinct parts: 1) text and 2) a
program. Text describes requirements, test expectation, and test results, and
A program automates running the protocol.

A goal of the Automated Test Framework (ATF) was to establish a common framework
for generating requirements, protocols and reports.  All text found in
doc/automated-test/boilerplate is common between all protocols.  The term
Test has been abstracted to mean any test type including: unit test,
verification test, validation test, integration test and tool validation test.

A validation package uses the following directory convention:

\begin{verbatim}
ToolName/
    validaton-pacakge.tex
        ../../boilerplate/document-layout.tex
        ../../boilerplate/tool-validation-package.tex
        intended-user.tex
        ../../boilerplate/protocol-overview.tex
        protocols/protocol-overview.tex
        protocols/protocol-automation.tex
        equipment.tex
        ../../boilerplate/conclusion.tex
        ..boilerplate/signed-test-results.tex
protocols/
    src/
        CheckInstall/
            Protocol.hs
            Protcol.tex
        ...
        Settings/
            Constants.hs
        Main.hs
    LICENSE
    protocol-automation.tex
    protocol-list.tex
    protocols.cabal
        Starterkit executable and starterkit library are produced.
    Setup.hs
    test-output/
\end{verbatim}
